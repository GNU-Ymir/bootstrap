

\usepackage[T1]{fontenc}
\usepackage{tgbonum}

\usepackage[lmargin=0.5in, rmargin=0.5in, paperwidth=29cm]{geometry}
\geometry{textwidth=23.1cm, textheight=24.50cm, marginratio={4:6,5:7}}


\usepackage{multicol}


\usepackage{framed,color,verbatim}

\usepackage[flushleft]{threeparttable}

\usepackage{minitoc}
\setcounter{minitocdepth}{5}


\usepackage{graphicx}
\usepackage{adjustbox}

\definecolor{shadecolor}{rgb}{.9, .9, .9}


\usepackage{fancyvrb}

\fvset{frame=single,framesep=1mm,fontfamily=courier,fontsize=\normalsize,numbers=left,framerule=.3mm,numbersep=1mm,commandchars=\\\{\}}

\usepackage[most]{tcolorbox}
\tcbset{
colback=gray!40,
colframe=white,
width=\dimexpr\textwidth\relax,
enlarge left by=0mm,
boxsep=5pt,
arc=0pt,outer arc=0pt,
}


\definecolor{applegreen}{rgb}{0.55, 0.71, 0.0}
\definecolor{antiquebrass}{rgb}{0.8, 0.58, 0.46}
\definecolor{PigBlue}{RGB}{14, 33, 160}
\definecolor{purple2}{RGB}{77, 60, 119}
\definecolor{background}{rgb}{1.0, 0.96, 0.89}

\usepackage{listings}
\lstdefinestyle{coloredverbatim}{
backgroundcolor=\color{background!40}, % Background color
numbers=left,                    % Position of line numbers
numberstyle=\tiny\color{red!35!white},   % Style of line numbers
numbersep=5pt,                   % Space between code and line numbers
xleftmargin=18pt,                % Left margin
xrightmargin=-2pt,
belowskip=10pt,
frame=single,                    % Frame around the code
framexleftmargin=15pt,           % Left margin for the frame
framexrightmargin=5pt,           % Right margin for the frame
breaklines=true,                 % Allow line breaks
showstringspaces=false,          % Don't show spaces in strings
linewidth=0.95\linewidth,            % Fixed width size
basicstyle=\ttfamily\footnotesize, % Font style and size
captionpos=t,                     % Caption position
literate={π}{{$\pi$}}1,
language=C,
keywordstyle=[1]\color{purple},       % keyword style
morekeywords=[1]{let, is, fn, unsafe, class, struct, in, throw, throws, match, over, pub, prv, prot, self, cte, mod, use, ref, catch, \_\_test}, % if you want to add more keywords to the set
keywordstyle=[2]\color{PigBlue!80},
keywords=[2]{i32, f32, i8, i16, i64, isize, u8, u16, u32, u64, usize, fsize, f32, f64, c8, c16, c32, Ok, Err, false, true},
keywordstyle=[3]\color{purple2!80!white},
keywords=[3]{expand, alias, copy, dcopy, mut, dmut, none, null, @union},
stringstyle = \color{black!30!applegreen}
}

\lstdefinestyle{intermediateVerb}{
backgroundcolor=\color{blue!10}, % Background color
numbers=left,                    % Position of line numbers
numberstyle=\tiny\color{purple!80!white},   % Style of line numbers
numbersep=5pt,                   % Space between code and line numbers
xleftmargin=18pt,                % Left margin
xrightmargin=-2pt,
belowskip=10pt,
frame=single,                    % Frame around the code
framexleftmargin=15pt,           % Left margin for the frame
framexrightmargin=5pt,           % Right margin for the frame
breaklines=true,                 % Allow line breaks
showstringspaces=false,          % Don't show spaces in strings
linewidth=0.95\linewidth,            % Fixed width size
basicstyle=\ttfamily\footnotesize, % Font style and size
captionpos=t,                     % Caption position
literate={π}{{$\pi$}}1,
language=C,
keywordstyle=[1]\color{purple},       % keyword style
morekeywords=[1]{let, is, fn, unsafe, class, struct, in, throw, throws, match, over, pub, prv, prot, self, cte, mod, use, catch, \_\_test, ref}, % if you want to add more keywords to the set
keywordstyle=[2]\color{PigBlue!100},
keywords=[2]{i32, f32, i8, i16, i64, isize, u8, u16, u32, u64, usize, fsize, f32, f64, c8, c16, c32, Ok, Err, false, true},
keywordstyle=[3]\color{purple!70!white},
keywords=[3]{expand, alias, copy, dcopy, mut, dmut, none, null, @union},
stringstyle = \color{black!30!applegreen}
}

\DeclareUnicodeCharacter{2320}{/}
\DeclareUnicodeCharacter{23AE}{|}
\DeclareUnicodeCharacter{2321}{/}
\DeclareUnicodeCharacter{2572}{\textbackslash}
\DeclareUnicodeCharacter{2571}{/}
\DeclareUnicodeCharacter{3C6}{$\phi$}
\DeclareUnicodeCharacter{2080}{${}_0$}
\DeclareUnicodeCharacter{3C0}{$\pi$}

\makeatletter
\renewcommand\tableofcontents{%
\@starttoc{toc}%
}
\makeatother


\usepackage{graphicx}
\usepackage{longtable}
\usepackage{wrapfig}
\usepackage{rotating}
\usepackage[normalem]{ulem}
\usepackage{amsmath}
\usepackage{amssymb}
\usepackage{capt-of}
\usepackage{hyperref}

\tikzstyle{every picture}+=[remember picture]
\tikzstyle{mProjectPP}=[circle,fill=yellow!70, minimum size=20pt,inner sep=0pt]
\tikzstyle{label}=[rectangle, fill=white, minimum size=40pt, inner sep=0pt]
\tikzstyle{Box}=[rectangle, fill=white, minimum size=40pt, inner sep=0pt]
\tikzstyle{Box2}=[rectangle, fill=white, minimum size=20pt, inner sep=0pt]


\usepackage{caption}
\usepackage{dblfloatfix}


\usepackage{makecell}


\setlength{\columnseprule}{0.01pt}
\def\columnseprulecolor{\color{black!10}}

\usetikzlibrary {arrows.meta, fit, calc, arrows, positioning}

\DeclareCaptionFormat{listing}{\rule{\dimexpr\linewidth-5pt\relax}{0.9pt}\par#1#2#3}
%% \DeclareCaptionFormat{listing}{#1#2#3}
\captionsetup[lstlisting]{format=listing,singlelinecheck=false, margin=0pt, font={sf}}
%\renewcommand\lstlistingname{Example}

\usepackage{awesomebox}
\usepackage{subcaption}

\usetikzlibrary{chains,shadows.blur}
