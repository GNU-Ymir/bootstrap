\section{Preamble}
\label{sec:preamble_type_and_values}

Ymir is an imperative programming language. Imperative programming is a
programming paradigm that uses statements to define the behavior the programm
will adopt at a given stage to change the memory state, and which statements
will be executed after. A value is a fundamental piece of data that is stored in
memory and that a variable can hold, represent, or manipulate. Values can be of
various types, such as integers, floating-point numbers, characters, strings,
booleans, and more complex structures like arrays, lists, objects, or
user-defined types. Values are the data manipulated by operators, expressions,
and statements within a program to perform computations and achieve the desired
outcomes.

\section{Memory representation}
\label{sec:memory_representation}

A program created by the Ymir compiler follows the stack and heap memory
management logic. This approach is commonly used by low-level programs, as it
represents the fundamental memory management logic typically available at the
user level in a modern operating system.

In such memory management system, the memory available to the program is divided in segments, that have different purposes


 \subsubsection*{Stack}
