The developement of a program using the Ymir language follows three fundamental
steps. First, a text editor is used to create text files named source files, in
which source code following the Ymir grammar and semantic is written. These
source files are passed to the compiler that transform source code information
into object files that is made up of machine code. These object files are
finally linked by a linker to form a an executable file. The second and third
step are generally made automatically when invoking the compiler. A source file
is text file encoded in \textit{utf-8} with the extension \textit{.yr}. It is
important to use the correct extension for the source file, as the compiler will
use it to determine in which language the source code is written. Indeed, the
Ymir language was made to be interoperable with the C language, and therefore
the compiler is also able to compile C source files, which are using
the \textit{.c} extension.

If the source code does not respect the syntax or semantic of the Ymir language,
the compiler uppon invokation, will issue error messages. These messages are
describing the origin of the errors, and are made to be readable by the
programmer to help them correct the problems.

\vspace{-20pt}%
\section{A first program}

A Ymir program is composed of symbols of different type. The first type of
symbols that one needs to know when writting their first program is function.
Indeed, a ymir program is always composed of at least one function that is
called when the program is executed. This function is named \token{main}. In the
following listing, a simple program that write \textit{Hello World!} to the
console is described. This program is composed of the function \token{main}
introduced by the token \token {fn}. This main function calls the function
\token{println}, which is described in the standard library, and used
to display texts to the console. The parentheses following the \token{println}
function name, are used to list parameters to pass to the function, here a
string literal. Note that statements are followed by a semicolon. The first line
of the example program starts with the keyword \token{use}. This line is used to
describe that a function is being used from the \token{std::io} module (here the
function \token{println}), which is automatically imported by the compiler since
it belongs to the standard library.

\begin{lstlisting}[style=coloredverbatim, caption=Source file \textit{hello.yr}]
use std::io;

fn main () {
  println ("Hello World!");
}
\end{lstlisting}

After calling the \token{std::io::println} function, the main function is exited
and with it the program.

\subsection{Compilation and execution}

Once we have a source file, the compiler has to be invoked to compile it and
create the executable file. The following listing present the command line to
execute in order to produce the \textit{hello} executable file, using the
\textit{hello.yr} source file.

\begin{lstlisting}[style=bashVerb, escapechar=@+]
@+\textcolor{teal!80}{alice@dev}@+:~$ gyc hello.yr -o hello
\end{lstlisting}

Once the compiler has finished, a new file named \textit{hello} will be present
in the current directory. This file is executable, and its execution will
display the text \textit{Hello World!}.

\begin{lstlisting}[style=bashVerb, escapechar=@+]
@+\textcolor{teal!80}{alice@dev}@+:~$ ./hello
Hello World!
\end{lstlisting}

\section{Structure of a simple Ymir program}

A ymir program can be composed of multiple symbols, for example multiple
functions. In the next example, three functions are described \token{main},
\token{foo} and \token{bar}. There is no need to define the symbols in a
particular order, however it is more common to put the \token{main} function at
the last position in the source file (or even put the main function in a
separate module).

\begin{lstlisting}[style=coloredverbatim, caption=First structured source file]
use std::io;

fn foo () {
  println ("Called foo");
  bar (); // calling the bar function
}

fn bar () {
  // because we use std::io, println can be called directly
  println ("Called bar");
}

/**
 * The entry point of the program
 */
fn main () {
  println ("Program started");
  foo ();
}
\end{lstlisting}

This example introduces the notion of comments. Comments are part of a source
file that are ignored by the compiler, and are only meant to be read by a human
and help them understand the code they are reading. We will see in a future
chapter, that comments can also be used to generate code documentation
automatically. A comments can be introduced using two different syntaxes:
\begin{itemize}
\item Everything following the token \token{//} until the end of the line, to
  create a single line comment.
\item Everyting enclosed within \token{/*} and \token{*/}, to create a multiline comment.
\end{itemize}

\subsection{Source code layout}

The compiler parses a source file by breaking its content into tokens, such as
parentheses, comas, brackets, etc. Tokens can be separated by an arbitry number
of white characters, that is spaces, line returns and tabulations. For example,
a function can be written in a single line, or by breaking into multiple lines
following a specific layout.


\begin{lstlisting}[style=coloredverbatim, caption=Arbitrary code layout example]
fn foo(){println("Called foo");bar();}

fn
main
(   )
{
println ("Program started");
foo();
}
\end{lstlisting}

\subsection{Prefered layout}

Because a strange layout can make the source code difficult to read, a standard
layout is generally prefered. In this layout, only one statement must be defined
in a line, and statements must be aligned according to their scope depth (a
scope being defined by the token \token{\{\}}). Spaces surrounds the operators,
and parameters operators such as \token{()} and \token{[]} are always spaced
as following \token{foo (a, b, c)}.

All the source codes presented in this book respect this layout, and it is
strongly encouraged to use the same layout when writting ymir programs, in order
to keep everything uniform.

\section{YIL language}

Ymir is a high level programming language, meaning that a statement in Ymir can
represent multiple low level instructions. C programming language is closer to a
machine level representation (even if there are still some high level
instructions that can be found in that language). In this book, in order to
fully understand the behavior of the Ymir language when executed on a machine,
and the memory representation a source code is describing, low level source code
close to the C language will be used. These low level source code are using a
language named YIL (Ymir Intermediate Language), and are a direct representation
of the high level source code into a low level one. A middle level
representation will be also used. In this middle level, some higher level
operations are still presents, such as loops, or some high level memory
management. In general it won't be necessary to distinguish between the two
level representations, as they won't be used at the same time, however if
required, L-YIL and M-YIL terms for respectively \textit{low}, and
\textit{middle} YIL, will be used.

In this book, Ymir language source code are presented in listing whose
background are yellow, M-YIL source code are presented in listings whose
background are teal, and L-YIL in listings whose background are green.

\begin{lstlisting}[style=coloredverbatim, caption=Source file \textit{hello.yr}, label=lst:hello_world]
use std::io;

fn main () {
    println ("Hello World!");
}
\end{lstlisting}

As an example, the source code presented in the above listing is transformed by
the compiler into the following M-YIL and L-YIL representations. The contents of
these representations are not yet discussed, but will be presented in the course
of this document.

\begin{lstlisting}[style=myilVerb, caption=M-YIL reprensentation of \textit{hello.yr}]
frame : [weak] std::io::println!{[c8]}::println (a(#8h) : [c8])-> void {
    std::io::print (a(#8h));
    std::io::putchar ('\n'c8);
    <unit-value>
}
frame : hello::main ()-> void {
    std::io::println!{[c8]}::println ("Hello World!"s8);
    <unit-value>
}
\end{lstlisting}

\begin{lstlisting}[style=lyilVerb, caption=L-YIL reprensentation of \textit{hello.yr}]
frame :  [weak] _Y3std2io7printlnNS2c8FS2c8Zv (let a(#1) : (len-> u64, ptr-> *(u8)))-> void {
    _Y3std2io5printFS2c8Zv(a(#1));
    putchar(10);
}
frame :  _Y5hello4mainFZv ()-> void {
    _Y3std2io7printlnNS2c8FS2c8Zv((len-> 12, ptr-> "Hello World!"));
}
frame :  main (let argc(#1) : u32, let argv(#2) : *(void))-> i32 {
    _yrt_run_main(argc(#1), argc(#2), &_Y5hello4mainFZv);
    return 0;
}
\end{lstlisting}


YIL representation can be generated using the option \textit{-fdump-ymir} and
compiling the source file with the Ymir compiler. This option generates a set of
intermediate representations, among which two files with the extensions
\textit{.ydump-sem} and \textit{.ydump-yil} can be found, containing the M-YIL and L-YIL reprensentations of the compiled
source file.

\begin{lstlisting}[style=bashVerb, escapechar=@+]
@+\textcolor{teal!80}{alice@dev}@+:~$ gyc -fdump-ymir hello.yr -o hello
@+\textcolor{teal!80}{alice@dev}@+:~$ ls
@+\textcolor{green!70!black}{hello}@+
hello.yr
hello.yr.ydump-decls.1
hello.yr.ydump-decls.2
hello.yr.ydump-opt
hello.yr.ydump-sem
hello.yr.ydump-yil
\end{lstlisting}

\section{Exercises}
\subsection{Exercise 1}

Write a Ymir program that prints the following to the console output :
\begin{verbatim}
Hello!
I wrote my first Ymir program.
Ain't that fun?
\end{verbatim}

\subsection{Exercise 2}
The following program contains several errors:

\begin{lstlisting}[style=bashVerb, caption=Ymir program with errors]
*/ Comments of the main function /*
fn main {
   println ("If this text")
   println ("appears on the screen, ');
   println ('you got it right').
)
\end{lstlisting}
Correct the errors and compile the program to test your corrections.

\subsection{Exercise 3}

What does the following program output to the screen?
\begin{lstlisting}[style=coloredverbatim]
use std::io;

fn pause() {
   print ("BREAK");
}

fn main () {
   println ("\nDear reader, ");
   print ("have a ");
   pause();
   println ("!");
}
\end{lstlisting}

\vfill%
\pagebreak
\section{Solutions}
\subsection{Exercise 1}

\begin{lstlisting}[style=coloredverbatim, caption=Solution for exercise 1]
use std::io;

fn main () {
  println ("Hello!");
  println ("I wrote my first Ymir program.");
  println ("Ain't that fun?");
}
\end{lstlisting}

\subsection{Exercise 2}

\begin{lstlisting}[style=coloredverbatimCorrect, escapechar=@]
@\hcb{\color{purple}{use} \color{black}{std::io;}}@

@\hcb{/*}@ @\textit{Comments of the main function}@ @\hcb{*/}@
fn main () {
   println (@\color{black!30!applegreen}{"If this text"}@)@\hcb{;}@
   println (@\color{black!30!applegreen}{"appears on the screen,}@ @\hcb{"}@);
   println (@\hcb{"}@@\color{black!30!applegreen}{you got it right}@@\hcb{"}@);
@\hcb{\}}@
\end{lstlisting}

\subsection{Exercise 3}
The output starts with a line return.
\begin{lstlisting}[style=bashVerb]

Dear reader,
have a BREAK!
\end{lstlisting}
