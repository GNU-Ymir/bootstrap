\chapter{Fundamental types, constants and variables}
\nopagecolor{}
\label{chap:prog_basics}

This chapter introduces the very basic programming concept needed when
programming with an imperative language. An imperative language consists of
constants and variables that hold data of a particular type. This data is used
to define the behavior of a program through a list of statements. Basic
statements can be divided into two distinct subsets, declarations and
instructions. Instructions either change the contents of the data contained in
variables or change the global state of the program to make it progress. A
program written in an imperative language can be thought of as a recipe, where
instructions are executed one at a time.

\minitoc%
\section{Constant and variables}

Within a program, declared symbols are identified by a name. Names (also called
identifiers) follow specific grammar rules and are case-sensitive (i.e., upper
and lower case are important). Identifiers can contain numbers, letters, and
underscores, but they must contain at least one letter, and numbers are always
preceded by at least one letter. An identifier can starts with an arbitry number
of underscores. For example, the following identifier is valid \token{\_\_id1\_\_},
but the identifier \token{\_\_1id\_\_} is not.

\begin{center}
\notebox{
  \colorbox{gray!10!white}{
    \begin{minipage}{0.9\linewidth}
    The formal grammar of a identifier is the following: \token{(\_)* [a-Z]([a-Z]|[0-9]|\_)*}
    \end{minipage}
  }
}
\end{center}

Some names that respect the grammar of identifier cannot be used as identifier,
as they are reserved by the language to define specific statements. These
identifiers are called keywords, and the complete list of reserved keywords is
given below. The meaning of these keywords will be explained through the course
of this book. Keyword list: \token{alias}, \token{assert}, \token{atomic},
\token{break}, \token{cast}, \token{catch}, \token{class}, \token{const},
\token{copy}, \token{cte}, \token{dcopy}, \token{def}, \token{dg}, \token{dmut},
\token{do}, \token{else}, \token{enum}, \token{expand}, \token{extern},
\token{false}, \token{for}, \token{fn}, \token{future}, \token{if},
\token{impl}, \token{in}, \token{is}, \token{lazy}, \token{let}, \token{loop},
\token{macro}, \token{match}, \token{mod}, \token{move}, \token{mut},
\token{none}, \token{null}, \token{of}, \token{panic}, \token{\_\_pragma},
\token{record}, \token{union}, \token{ref}, \token{return}, \token{spawn},
\token{static}, \token{entity}, \token{throw}, \token{throws}, \token{trait},
\token{true}, \token{typeof}, \token{\_\_test}, \token{unsafe}, \token{use},
\token{\_\_version}, \token{while}, \token{with}.

\vfill%

