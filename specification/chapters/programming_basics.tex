\chapter{Fundamental types, constants and variables}
\nopagecolor{}
\label{chap:prog_basics}

This chapter introduces the very basic programming concept needed when
programming with an imperative language. An imperative language consists of
constants and variables that hold data of a particular type. This data is used
to define the behavior of a program through a list of statements. Basic
statements can be divided into two distinct subsets, declarations and
instructions. Instructions either change the contents of the data contained in
variables or change the global state of the program to make it progress. A
program written in an imperative language can be thought of as a recipe, where
instructions are executed one at a time.

\minitoc%
\chapter{Fundamental types, constants and variables}
\nopagecolor{}
\label{chap:prog_basics}

This chapter introduces the very basic programming concept needed when
programming with an imperative language. An imperative language consists of
constants and variables that hold data of a particular type. This data is used
to define the behavior of a program through a list of statements. Basic
statements can be divided into two distinct subsets, declarations and
instructions. Instructions either change the contents of the data contained in
variables or change the global state of the program to make it progress. A
program written in an imperative language can be thought of as a recipe, where
instructions are executed one at a time.

\minitoc%
\chapter{Fundamental types, constants and variables}
\nopagecolor{}
\label{chap:prog_basics}

This chapter introduces the very basic programming concept needed when
programming with an imperative language. An imperative language consists of
constants and variables that hold data of a particular type. This data is used
to define the behavior of a program through a list of statements. Basic
statements can be divided into two distinct subsets, declarations and
instructions. Instructions either change the contents of the data contained in
variables or change the global state of the program to make it progress. A
program written in an imperative language can be thought of as a recipe, where
instructions are executed one at a time.

\minitoc%
\chapter{Fundamental types, constants and variables}
\nopagecolor{}
\label{chap:prog_basics}

This chapter introduces the very basic programming concept needed when
programming with an imperative language. An imperative language consists of
constants and variables that hold data of a particular type. This data is used
to define the behavior of a program through a list of statements. Basic
statements can be divided into two distinct subsets, declarations and
instructions. Instructions either change the contents of the data contained in
variables or change the global state of the program to make it progress. A
program written in an imperative language can be thought of as a recipe, where
instructions are executed one at a time.

\minitoc%
\input{parts/programming_basics}



