\section{Floating point types}


Numbers with a decimal point are called floating point numbers in Ymir. Unlike
integers, floating-point numbers have a predetermined precision that depends on
the number of bytes used to represent them. There are four floating-point types:
\token{f32}, \token{f64}, \token{f80} and \token{fsize}. As with integer types,
the number following the letter \token{f} defines the number of bits in memory
used to represent a value of that type, and \token{fsize} is an
architecture-dependent type that takes the largest floating point representation
available on the machine. The accuracy of a floating-point defines the capacity
of a program to distinguish between two floating-point values. Indeed, since
there are an infinite number of values between tow decimal values, it is
impossible to represent them all with a finite number of bits. This behavior is
called stepping, and it is important to keep in mind that a number might not be
representable by a floating point value.

\begin{lstlisting}[style=coloredVerbatim, label=lst:(chap2):stepping_example, caption=Example of floating point stepping]
let z = 1.989328f; // 'f' suffix to create a f32 literal
let mut x = 0.98769778678f;
let mut y = z - x;

let w = x + y;

println ("(X : ", x, ", Y : ", y, ")");
println ("(Z : ", z, ", W : ", w, ")");
println ("Z == W : ", z == w);
\end{lstlisting}
\vspace{-10pt}%
\begin{lstlisting}[style=bashVerb]
(X : 0.987698, Y : 1.00163)
(Z : 1.98933, W : 1.98933)
Z == W : false
\end{lstlisting}

Listing~\ref{lst:(chap2):stepping_example} shows an example of stepping. In this
example, even though the opposite operation is performed to create the value in
the \token{w} variable from the value in the \token{y} variable, due to
stepping, the result of the operation appears to create different values, and
therefore \token{z} is different from \token{w}. In this example, \token{f32}
values are used; replacing them with \token{f64} values eliminates the stepping
problem (for this particular example. Floating-point values coded in 64 bits
have better accuracy than 32 bits, but are still subject to stepping.

Floating point literal values consist of two integer parts separated by the
\token{.} token, with no spaces. As with integer literal values, the \token{\_}
token can be used to separate long literal values and make them more readable.
By default, a floating-point literal is of type \token{f64}, but it can be
prefixed with the letters \token{f}, \token{d}, \token{l}, and \token{r} for the
types \token{f32}, \token{f64}, \token{f80}, and \token{fsize}, respectively.

\mynotebox{
  Advanced literal representations are available, such as hexadecimal,
  scientific notation, etc. These are described in the advanced chapter on
  floating-point types, cf. Chapter~\ref{sec:(chap3):floats}.
}
