\vfill\pagebreak
\section{Boolean type and Boolean algebra}

A Boolean is a fundamental data type that can hold one of two values,
\token{true} or \token{false}. It is designed to represent truth values in
logical operations and Boolean algebra. Boolean values form the foundation of
conditional processing, enabling programs to follow different branches of
execution depending on control flow decisions. In Ymir the Boolean type is
represeted by the keyword \token{bool}, and is encoded in one byte of memory
where the value \token{0b00000000} represents the value \token{false}, and
\token{true} is represented by any other value.

\subsection{Boolean algebra}

Boolean algebra is concerned with truth values and the operations used to
combine them. It is built on three fundamental operations, AND, OR, and NOT (in
contrast to the four basic operations of numerical algebra — addition,
subtraction, multiplication, and division). Any other logical operation can be
expressed in terms of these three. Mastering Boolean algebra is essential for
computer science, as it forms the foundation of programming and control flow.

The AND operation (also called conjunction), denoted as $\land$ in Boolean
algebra and written as \token{&&} in Ymir, compares two Boolean values and
returns \token{true} only if both operands are \token{true}. The OR operation
(also called disjunction), denoted as $\lnot$ in Boolean algebra and written as
\token{||} in Ymir, also takes two operands and returns \token{true} if at least
one of them is \token{true}, regardless of the other. The final elementary
operator, NOT (also called negation), denoted as $\lnot$ in Boolean algebra and
written as \token{!} in Ymir, takes a single operand and inverts its value,
turning \token{true} into \token{false} and \token{false} into \token{true}.

