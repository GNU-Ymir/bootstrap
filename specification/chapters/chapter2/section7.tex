\vfill\pagebreak
\section{Boolean type and Boolean algebra}

A Boolean is a fundamental data type that can hold one of two values,
\token{true} or \token{false}. It is designed to represent truth values in
logical operations and Boolean algebra. Boolean values form the foundation of
conditional processing, enabling programs to follow different branches of
execution depending on control flow decisions. In Ymir the Boolean type is
represeted by the keyword \token{bool}, and is encoded in one byte of memory
where the value \token{0b00000000} represents the value \token{false}, and
\token{true} is represented by any other value.

\subsection{Primary operators}

Boolean algebra is concerned with truth values and the operations used to
combine them. It is built on three fundamental operations, AND, OR, and NOT (in
contrast to the four basic operations of numerical algebra — addition,
subtraction, multiplication, and division). Any other logical operation can be
expressed in terms of these three. Mastering Boolean algebra is essential for
computer science, as it forms the foundation of programming and control flow.

The AND operation (also called conjunction), denoted as $\land$ in Boolean
algebra and written as \token{&&} in Ymir, compares two Boolean values and
returns \token{true} only if both operands are \token{true}. The OR operation
(also called disjunction), denoted as $\lnot$ in Boolean algebra and written as
\token{||} in Ymir, also takes two operands and returns \token{true} if at least
one of them is \token{true}, regardless of the other. The final elementary
operator, NOT (also called negation), denoted as $\lnot$ in Boolean algebra and
written as \token{!} in Ymir, takes a single operand and inverts its value,
turning \token{true} into \token{false} and \token{false} into \token{true}. The
Boolean value \token{false} is commonly denoted by \bot, while \token{true} is
represented by \top.

\begin{table}[h]
  \centering
  \scalebox{0.9}{
  \begin{minipage}[t]{0.4\textwidth}
    \centering
    \begin{tabular}{p{.15\textwidth}|p{.15\textwidth}|p{.2\textwidth}|p{0.2\textwidth}}
      \toprule[0.6pt]
      $x$ & $y$ & $x \land y$ & $x \lor y$ \\
      \toprule[0.6pt]        
      \top & \top & \top & \top \\
      \top & \bot & \bot & \top \\
      \bot & \top & \bot & \top \\
      \bot & \bot & \bot & \bot \\
      \midrule[0.2pt]
    \end{tabular}
    \captionof{table}{AND, OR operators}
  \end{minipage}%
  \hspace{50pt}%
  \begin{minipage}[t]{0.3\textwidth}
    \centering
    \begin{tabular}{p{.15\textwidth}|p{.30\textwidth}}
      \toprule[0.6pt]
      $x$ & $\lnot x$ \\
      \toprule[0.6pt]        
      \top & \bot \\
      \bot & \top \\
      \midrule[0.2pt]
    \end{tabular}
    \captionof{table}{NOT operator}
  \end{minipage}
  }
%  \caption{\label{tab:(chap2):boolean_operators} Boolean algebra operators}
\end{table}


\vspace{-10pt}%
\subsection{Secondary operators}

Boolean algreba defines secondary operators, that are formed from the basic
operations. One can list three of them, implication, bidirectional implication,
and exclusive disjunction. All secondary operations can be transformed into a
expression of basic operators.

\begin{itemize}
\vspace{-5pt}%
\item Implication: $x \rightarrow y = \lnot x \lor y$.
\vspace{-5pt}%
\item Bidirectional implication : $x \leftrightarrow y = (\lnot x \lor y) \land (x \lor \lnot y)$ 
\vspace{-5pt}
\item Exclusive disjunction (or XOR) : $x \oplus y = (x \land \lnot y) \lor (\lnot x \land y)$
\end{itemize}
\vspace{-5pt}%

\begin{table}[h]
  \centering
  \scalebox{0.9}{
    \begin{tabular}{p{.15\textwidth}|p{.15\textwidth}|p{.2\textwidth}|p{0.2\textwidth}|p{0.2\textwidth}}
      \toprule[0.6pt]
      $x$ & $y$ & $x \rightarrow y$ & $x \leftrightarrow y$ & $x \oplus y$ \\
      \toprule[0.6pt]        
      \top & \top & \top & \top & \bot \\
      \top & \bot & \bot & \bot & \top \\
      \bot & \top & \top & \bot & \top \\
      \bot & \bot & \top & \top & \bot \\
      \midrule[0.2pt]
    \end{tabular}
  }
  \caption{\label{tab:(chap2):boolean_operators} Secondary operators}
\end{table}




To understand rewriting, we can examine the following table, which presents the
truth values of the three operators. As an example, consider the implication
operator. The two simplest cases occur when the value of $x$ is true: in this
situation, for the implication to evaluate to true, $y$ must also be true.
Conversely, when the value of $x$ is false, the implication provides no
constraint on $y$. Indeed, since implication is one‑sided, the truth of $y$ does
not determine the truth of $x$, and $y$ may therefore be either true or false.
Although in everyday language the expression $\lnot x \lor y$ may seem to carry
a different meaning, it produces exactly the same truth table.

Ymir does not provide secondary operators for Boolean values, since the primary
operators already perform all the necessary operations on them.

\subsection{Laws on Boolean operators}
\subsection{Specificities of Boolean operators in Ymir}
