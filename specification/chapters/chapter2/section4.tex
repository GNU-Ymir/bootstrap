\section{Integer types}

The types \token{i8}, \token{i16}, \token{i32}, \token{i64} and \token{isize}
are available to store integer values. They differ in their value range, which
is determined by the number of bits they use in memory to store values. The
\token{isize} type is the largest integer type the system can use (e.g. 64 bits
on arm64 or 32 bits on x86). Integer types starting with the letter \textit{i}
refer to integer values with a sign, in contrast to integer types starting with
the letter \textit{u}, the types \token{u8}, \token{u16}, \token{u32},
\token{u64} and \token{usize}. The sign refers to the possibility for the values
encoded by these types to be negative, which affects the range of values that
can be encoded by them.

Integer literals are formed using a decimal form that contains only numbers and
no spaces. A literal is a word (or list of words) in the source file that
describes a constant value of a particular type. We have seen examples of
integer literals in the previous listings, such as \token{12}, \token{36},
\ldots. The \token{\_} token can be used to separate long integer literals (e.g.
\token{1\_000\_000\_000\_000}). Integer literals can be suffixed with the name
of the integer type for which we want to create a value. In fact, by default, a
value of type \token{i32} is created, but it may be useful to create literals of
other integer types (e.g., \token{12u8}).

\mynotebox{
  More advanced literals using octal, hexadecimal and binary
  representation can be used. These are described in the advanced chapter
  on integers, cf. Chapter~\ref{sec:(chap3):integers}.
}
