\section{Preamble}%
\label{sec:preamble_compound_types}

Before presenting the specifications of the different compound types, let's
define some useful language elements.

\begin{itemize}
\item \textit{Data borrowing} -- A data borrowing value is a value that points
  to data that is not automatically copied when the value is copied. The
  simplest example of data borrowing is the use of a pointer type, where the
  value of the pointer provides information about a memory address containing
  data. The data that the pointer points to is not copied when the pointer is
  copied. An illustration of this is given in figure~\ref{fig:(chap4):data_borrowing}.
  In this example, both variables \token{x} and \token{y} borrow the data
  located on the heap at address \token{0xfa45b987}. Copying the value of the
  variable \token{y} to another variable does not make a copy of the borrowed
  value (which is on the heap), but only copies the value directly inside
  \token{y}, which in this case is on the stack. The borrowed data can be on
  the stack or on the heap, and the same can be said for the borrowing data.
  Such memory is called ``borrowed'' because it can live longer than the value
  pointing to it, and the pointer only temporarily borrows its value for reading
  or writing.

  In the case of borrowing data, mutability is crucial because multiple variables
  refer to the same segment of memory. This can lead to unpredictable side
  effects. Memory mutability is used to manage such behaviour and to ensure that
  only a few variables have mutable access to values. Most importantly, it is a
  safeguard against the accidental creation of mutable memory references.


\end{itemize}

\begin{figure}[H]
  \centering
  \begin{adjustbox}{max size=\linewidth}
    \begin{tikzpicture}
      % ================ Axis ========================
      \draw (0, 0) -- coordinate (LEFT) (0, 0);
      \draw[-Stealth](0, 10) -- coordinate (YAx) (0, 1.5);
      \draw[-Stealth](7.2, 10) -- coordinate (YAx) (7.2, 1.5);
      \draw[Box] (3.5, 10.7) -- (3.5, 10.7) node[anchor=center]{\textbf{STACK}};

      \filldraw (0, 6.5) circle (1pt) node[align=center, left] {+0};
      \filldraw (0, 5.5) circle (1pt) node[align=center, left] {+8};
      \filldraw (0, 4.5) circle (1pt) node[align=center, left] {+16};
      \filldraw (0, 3.5) circle (1pt) node[align=center, left] {+24};
      \filldraw (0, 2.5) circle (1pt) node[align=center, left] {+32};

      %% % ================ Foo ========================
      \fill[black!10] (0.2, 10) rectangle (7, 8);
      \draw[Box] (3.5, 7.6) -- (3.5, 7.6) node[anchor=center]{\textbf{.}};
      \draw[Box] (3.5, 7.4) -- (3.5, 7.4) node[anchor=center]{\textbf{.}};
      \draw[Box] (3.5, 7.2) -- (3.5, 7.2) node[anchor=center]{\textbf{.}};

      \draw[line width=1pt] (0.18, 6.9) rectangle (7.02, 2.48);
      \fill[olive!10] (0.2, 6.5) rectangle (7, 2.5);
      \draw[Box] (3.5, 6.7) -- (3.5, 6.7) node[anchor=center]{\emph{foo}};

      \draw[Box] (0.5, 5.5) -- (0.5, 5.5) node[anchor=center]{\textbf{\textit{x}}};
      \draw[line width=0.005pt] (0.3, 6.4) rectangle (6.8, 4.6);
      \draw[line width=0.005pt] (0.7, 5.5) rectangle (6.8, 5.5);

      \draw[Box] (3.5, 6) -- (3.5, 6) node[anchor=center]{\emph{len~=}~$3$};
      \draw[Box] (3.5, 5) -- (3.5, 5) node[anchor=center]{\emph{ptr~=}~0xfa45b987};

      \draw[Box] (0.5, 3.5) -- (0.5, 3.5) node[anchor=center]{\textbf{\textit{y}}};
      \draw[line width=0.005pt] (0.3, 4.4) rectangle (6.8, 2.6);
      \draw[line width=0.005pt] (0.7, 3.5) rectangle (6.8, 3.5);

      \draw[Box] (3.5, 4) -- (3.5, 4) node[anchor=center]{\emph{len~=}~$3$};
      \draw[Box] (3.5, 3) -- (3.5, 3) node[anchor=center]{\emph{ptr~=}~0xfa45b987};


      %% % ================ Heap ========================

      \draw[-Stealth](10, 10) -- coordinate (YAx) (10, 1.5);
      \draw[-Stealth](17.2, 10) -- coordinate (YAx) (17.2, 1.5);
      \filldraw (17.2, 7.5) circle (1pt) node[align=center, right] {0xfa45b987};

      % ================ Data ========================
      \draw[Box] (13.5, 10.7) -- (13.5, 10.7) node[anchor=center]{\textbf{HEAP}};
      \fill[black!10] (10.2, 10) rectangle (17, 8);
      \fill[teal!20] (10.2, 7.9) rectangle (17, 5);
      \draw[line width=0.005pt] (10.4, 7) rectangle (16.8, 7);
      \draw[line width=0.005pt] (10.4, 6) rectangle (16.8, 6);

      \draw[Box] (13.5, 7.5)  (13.8, 7.5) node[anchor=center]{$1$};
      \draw[Box] (13.5, 6.5)  (13.8, 6.5) node[anchor=center]{$2$};
      \draw[Box] (13.5, 5.5)  (13.8, 5.5) node[anchor=center]{$3$};

      \draw[thick,-Stealth,shorten >=1pt] (7.3, 5) to [out=0,in=180] node[below left, yshift=3mm] {} (9.9, 7.5);

      \draw[thick,-Stealth,shorten >=1pt] (7.3, 3) to [out=0,in=180] node[below left, yshift=3mm] {} (9.9, 7.5);

    \end{tikzpicture}
  \end{adjustbox}
  \vspace{-30pt}%
  \caption{\label{fig:data_borrowing}Example of data borrowing}
\end{figure}
%
%% \vspace{2pt}%


\begin{itemize}
  \setlength\itemsep{-4pt}
\item \textit{Data movement} -- Data movement is the process of copying a value
  from one segment of memory to another. This can be done by allocation, memory
  copy (or deep copy), function calls and other operations. Data movements are
  always explicit, which ensures that no unwanted copies are made that could
  potentially slow down the program.

\item \textit{lvalue} -- An lvalue refers to the left operand in a \textit{data
  movement} operation. In other words, it is an expression that refers to a
  segment of memory that is modified as a result of the \textit{data movement}.
  An lvalue can take various forms, including a simple variable, a function
  parameter, an index operation, and more.

\item \textit{rvalue} -- An rvalue is the right operand involved in a
  \textit{data movement} operation. A rvalue can be aliased (using the keyword
  \token{alias}), referenced (keyword \token{ref}), copied with
  (\token{copy}, \token{dcopy}), or it can be implicit. Implicit means that
  the \textit{data movement} does not borrow mutable data and can therefore be
  allowed implicitly. In the case of an implicit \textit{data movement} there is
  no copying of borrowed data and there is no new borrowing of mutable memory.

\end{itemize}






