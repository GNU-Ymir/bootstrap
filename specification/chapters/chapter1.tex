\chapter{Fundamentals}
\nopagecolor{}
\label{chap:chap1}

This chapters presents the fundamental characteristic of the Ymir programming language. It introduces the first necessary steps to build a Ymir program, and is also a first step into the basics of low level programming via a high level programming language. In this chapter is also presented the basics on how this book is structured and will guide the reader into fully understanding the concepts presented in this book.

\minitoc%

The developement of a program using the Ymir language follows three fundamental
steps. First, a text editor is used to create text files named source files, in
which source code following the Ymir grammar and semantic is written. These
source files are passed to the compiler that transform source code information
into object files that is made up of machine code. These object files are
finally linked by a linker to form a an executable file. The second and third
step are generally made automatically when invoking the compiler. A source file
is a text file encoded in \textit{utf-8} with the extension \textit{.yr}. It is
important to use the correct extension for the source file, as the compiler will
use it to determine in which language the source code is written. Indeed, the
Ymir language was made to be interoperable with the C language, and therefore
the compiler is also able to compile C source files, which are using the
\textit{.c} extension.


\begin{figure}[h]
  \vspace{-5pt}%
  \centering
  \scalebox{0.90}{
    \begin{tikzpicture}

      \draw (0, 0) -- coordinate (X) (0, 0);
      \draw (4, 0) -- coordinate (Y) (4, 0);
      \draw (8, 0) -- coordinate (Z) (8, 0);
      \draw (12, 0) -- coordinate (W) (12, 0);

      \chainicon{0}{0}{3}{2}{4mm}{0.2}{0}

      %% \chain[orange]{-4}{0}{0.8} % orange, smaller
                                %
      %% \fileicon{0}{0}{main.yr}

      %% \fileicon{3.2}{0.2}{}
      %% \fileicon{3}{0}{utils.yr}

      %% \fileicon{6}{0}{lib.a}

      %% \gearicon[orange]{0}{0}{9}{0.8cm}{0.2cm}{17}{9}{1.8}      % 6 teeth, flat-top trapezoid

    \end{tikzpicture}
  }
  \caption{\label{fig:(chap1):compilation_chain} Steps of compilation to build an executable}
\end{figure}


If the source code does not respect the syntax or semantic of the Ymir language,
the compiler uppon invokation, will issue error messages. These messages are
describing the origin of the errors, and are made to be readable by the
programmer to help them correct the problems.

\section{A first program}

A Ymir program is composed of symbols of different type. The first type of
symbols that one needs to know when writting their first program is function.
Indeed, a ymir program is always composed of at least one function that is
called when the program is executed. This function is named \token{main}. In the
following listing, a simple program that write \textit{Hello World!} to the
console is described. This program is composed of the function \token{main}
introduced by the token \token {fn}. This main function calls the function
\token{println}, which is described in the standard library, and used to display
texts to the console. The parentheses following the \token{println} function
name, are used to list parameters to pass to the function, here a string
literal. Note that statements are followed by a semicolon. After calling the
\token{println} function, the main function is exited and with it the program.

The first line of the example program starts with the keyword \token{use}. This
line is used to describe that a function is being used from the \token{std::io}
module, here the function \token{println}. Standard library modules are
automatically available and can be used without requiring additional importation
step.

\begin{lstlisting}[style=coloredverbatim, caption=Source file \textit{hello.yr}]
use std::io;

fn main () {
  println ("Hello World!");
}
\end{lstlisting}


\subsection{Compilation and execution}

Once we have a source file, the compiler has to be invoked to compile it and
create the executable file. The following listing presents the command line to
execute in order to produce the \textit{hello} executable file, using the
\textit{hello.yr} source file.

\begin{lstlisting}[style=bashVerb, escapechar=@+]
@+\textcolor{teal!80}{alice@dev:\mtilde}@+$ gyc hello.yr -o hello
\end{lstlisting}

Once the compiler has finished, a new file named \textit{hello} will be present
in the current directory. This file is executable, and its execution will
display the text \textit{Hello World!}.

\begin{lstlisting}[style=bashVerb, escapechar=@+]
@+\textcolor{teal!80}{alice@dev:\mtilde}@+$ ./hello
Hello World!
\end{lstlisting}


\vfill
\pagebreak

\section{Variable declaration}

Within a function, a variable can be declared using the \token{let} keyword. A
variable is essentially an identifier that refers to a specific value. Each
value has an associated type, and this type determines how the program
interprets and operates on the variable. To visualize this, we can think of a
variable as a \textit{box} that contains the value. The size and shape of the
box depend on the variable’s type, ensuring that only values of the correct form
can be stored inside. In this chapter, we will focus on the basic types
available in Ymir. These types are limited to storing scalar values. The
fundamental categories are:

\vspace{-10pt}%
\begin{itemize}
\setlength\itemsep{0.2mm}
\item integer types,
\item floating-point types,
\item character types,
\item the boolean type.
\end{itemize}\vspace{-10pt}%

Although Ymir provides several variants of integers, floating-point numbers, and
characters, each capable of representing different ranges of values, we will
concentrate here on three representative examples: \token{i32} (an integer),
\token{f32} (a floating-point number), and \token{c8} (a character).

\smallskip{}
\noindent{}In the following program, four variables are declared: \token{x}, \token{y},
\token{z}, and \token{w}. Their respective types are \token{i32}, \token{f32},
\token{c8}, and \token{bool}.

\begin{lstlisting}[style=coloredVerbatim, caption=simple variable declaration, label=lst:(chap2):basic_variables]
use std::io;

fn main () {
  let x : i32 = 10;
  let y : f32 = 3.1415;
  let z : c8 = 'c';
  let w : bool = false;

  println ("X : ", x);
  println ("Y : ", y);
  println ("Z : ", z);
  println ("W : ", w);
}
\end{lstlisting}

\begin{figure}[h]
  \vspace{-5pt}%
  \centering
  \scalebox{0.90}{
    \begin{tikzpicture}

      \draw (0, 0) -- coordinate (X) (0, 0);
      \draw (4, 0) -- coordinate (Y) (4, 0);
      \draw (8, 0) -- coordinate (Z) (8, 0);
      \draw (12, 0) -- coordinate (W) (12, 0);

      \fill[olive!10] ($(X)$) rectangle ($(X)+(3, 5)$);
      \filldraw ($(X)+(1.5, -0.2)$) node [align=center, below] {\textit{x}};
      \filldraw ($(X)+(1.5, 2.5)$) node [align=center] {10};
      \filldraw ($(X)+(-0.4, 2.5)$) node [align=center, rotate=90] {\small {\textit{32 bits}}};
      \draw[Stealth-Stealth] ($(X)+(-0.1, 0.1)$) -- ($(X)+(-0.1, 4.9)$);

      \fill[olive!10] ($(Y)$) rectangle ($(Y)+(3, 5)$);
      \filldraw ($(Y)+(1.5, -0.2)$) node [align=center, below] {\textit{y}};
      \filldraw ($(Y)+(-0.4, 2.5)$) node [align=center, rotate=90] {\small {\textit{32 bits}}};
      \filldraw ($(Y)+(1.5, 2.5)$) node [align=center] {3.1415};
      \draw[Stealth-Stealth] ($(Y)+(-0.1, 0.1)$) -- ($(Y)+(-0.1, 4.9)$);

      \fill[olive!10] ($(Z)$) rectangle ($(Z)+(3, 1.25)$);
      \filldraw ($(Z)+(1.5, -0.2)$) node [align=center, below] {\textit{z}};
      \filldraw ($(Z)+(-0.4, 0.625)$) node [align=center, rotate=90] {\small {\textit{8 bits}}};
      \filldraw ($(Z)+(1.5, 0.625)$) node [align=center] {'c'};
      \draw[Stealth-Stealth] ($(Z)+(-0.1, 0.1)$) -- ($(Z)+(-0.1, 1.15)$);

      \fill[olive!10] ($(W)$) rectangle ($(W)+(3, 1.25)$);
      \filldraw ($(W)+(1.5, -0.2)$) node [align=center, below] {\textit{w}};
      \filldraw ($(W)+(1.5, 0.625)$) node [align=center] {false};
      \filldraw ($(W)+(-0.4, 0.625)$) node [align=center, rotate=90] {\small {\textit{8 bits}}};
      \draw[Stealth-Stealth] ($(W)+(-0.1, 0.1)$) -- ($(W)+(-0.1, 1.15)$);

    \end{tikzpicture}
  }
  \caption{\label{fig:(chap2):variable_bins} Representation of variables of Listing~\ref{lst:(chap2):basic_variables} as bins}
\end{figure}


To better understand how variables work, it can be helpful to imagine them as
boxes. This is only a mental image: in practice, the program does not literally
create boxes, but the metaphor captures the global idea of how values are stored
and accessed. When the program presented in
Listing~\ref{lst:(chap2):basic_variables}, is execute, and the \textit{main}
function is entered, four such \textit{boxes} are created, each containing a
value (see Figure~\ref{fig:(chap2):variable_bins}). The contents of these boxes
can be retrieved by referring to their labels (or identifier), as shown in the
\textit{println} statements from lines 9 to 12 in
Listing~\ref{lst:(chap2):basic_variables}.

\smallskip{}
\noindent{}Declaring a variable follows a specific syntax. After the keyword \token{let},
we may indicate the \textit{shape} of the box (that is, the type of value it can
contain) by writing it between the \token{:} and \token{=} symbols. This type
annotation is optional, because the compiler can usually determine the box’s
shape automatically from the first value placed inside. Once the box’s shape is
set, however, only values of that shape can be stored in it. In the example, the
declared boxes are sealed: their contents cannot be changed once assigned. These
are constants. To create a box that can be reopened and filled with new
contents, we use the keyword \token{mut} after \token{let}. Such boxes are
called mutable variables. For convenience, both sealed and unsealed
\textit{boxes} are often referred to as variables but technically only the
mutable ones behave as true variables in the usual sense.

\smallskip{}

The value stored in a mutable variable can be changed using the \token{=}
operator. This operator assigns a new value to a variable, and the assignment is
valid only if the new value (written on the right-hand side) has the same type
as the variable being updated (written on the left-hand side).

\smallskip{}
\noindent{}In the terminology of assignments, the left operand is called the
\textit{lvalue} (short for \textit{left value}), and the right operand is called
the \textit{rvalue} (short for \textit{right value}). In terms of the box
metaphor, the \textit{lvalue} represents the box itself, while the
\textit{rvalue} represents the new content being placed inside.
Listing~\ref{lst:(chap2):value_change} illustrates such a value modification,
where the variable \token{x} serves as the \textit{lvalue}. The resulting
program output is shown directly below the listing.

\begin{lstlisting}[style=coloredVerbatim, caption=Example of mutable variable, label=lst:(chap2):value_change]
use std::io;

fn main () {
  let mut x = 10;

  x = 42;
  println ("X : ", x);

  x = 314;
  println ("X : ", x);
}
\end{lstlisting}
\vspace{-10pt}%
\begin{lstlisting}[style=bashVerb, label=lst:(chap2):value_change_output]
X : 42
X : 314
\end{lstlisting}

To explain the behavior of the source code shown in
Listing~\ref{lst:(chap2):value_change}, we can imagine the execution of a
program as a cursor moving downward through the instructions, beginning with the
\token{main} function (the entry point of any Ymir program).

\smallskip{}
\noindent{}The program first creates a box for the variable \token{x} and places
the value \token{10} inside. As the cursor continues, it reaches the assignment
operation, which replaces the contents of the box with the new value \token{42}.
The cursor then moves to the next instruction, a call to the \textit{println}
function, which retrieves the value stored in \token{x} and displays it on the
console.

\smallskip{}
\noindent{}The instructions on lines 9 and 10 work in the same way: the cursor
moves downward, step by step, carrying out each operation in order. This style
of execution, where the program state is modified step by step as control flows
from one instruction to the next, is known as imperative programming, and more
specifically, procedural programming.


\smallskip{}
\noindent{}Ymir provides different ways to structure and execute programs, but
all of these approaches follow the imperative procedural paradigm. This
paradigm, which describes step-by-step execution of instructions, forms the
foundation of the language. Its principles and variations will be thoroughly
explored throughout this book.

\section{Unit tests}%
\label{sec:unit_test}

Unittest are functions that are not callable. They are defined using the keyword
\token{\_\_test} followed by a body expression. Unlike function, they don't
have identifier and do not take parameters. A test is failing if it throws an
exception, and succeeding if it exits normally.

\begin{lstlisting}[style=coloredverbatim]
__test {
    assert (false, "A test that fails.");
}

__test {
    assert (true, "A test that succeeds.");
}
\end{lstlisting}


Unittest are compiled with the option \token{-funittest}. This option produces
a executable file that launchs the unittest instead of running the function
\token{main}.

\begin{lstlisting}[style=bashVerb, escapechar=@+]
@+\textcolor{teal!80}{alice@dev}@+:~$ gyc -funittest main.yr
\end{lstlisting}

\vfill%
\pagebreak

\section {Ranges}%
\label{sec:range_type}

Range is a compound type consisting of four elements that describe a range of
values. The four elements are \token{fst} the first value of the range (e.g.
\token{0}), \token{scd} the last value of the range (e.g. \token{10}),
\token{step} the step of the range (e.g. \token{2}), and \token{contains} of
type bool, which indicates whether the last value \token{scd} is contained in
the range or not. There are only three types that can describe the inner
components of a range: integer, floating point and character. The type of the
range is defined by the inner type preceded by the token \token{..} (e.g.
\token{..i32} describes a range of \token{i32} values). Ranges are useful for
iteration or for accessing a subset of values (for example, a subset of a
slice).

\subsection {Literals}

Range literals are described using the \token{..} token or the \token{...}
token. The \token{..} token is used to define a range whose final value is not
included in the range, and the \token{...} token defines a range whose final
value is included. If different tokens are used to describe the literal, the
type is the same, and the \token{..} token is always the only token used to
describe a range type.

\begin{lstlisting}[style=coloredverbatim]
let a : ..i32 = 0 .. 2;
let b : ..i32 = 0 ... 2;

assert (a.fst == b.fst);
assert (a.scd == b.scd);
assert (a.step == b.step);
assert (!a.contain && b.contain);
\end{lstlisting}

Range values can be decreasing, in which case the step is negative. Note that
for ranges of unsigned integers and character values, it is theoretically
impossible to have a negative value for the step. However, there is a bit of
cheating going on here, using the overflow limitation of types to create a value
that, when added to \token{fst}, equals \token{fst - abs (step)} (in practice,
this is exactly the same as adding a negative value at the binary level, but it
is not really the valid high level representation). For this reason it can be
considered that step is always a signed version of the type, even if the field
type is considered to be the same as the type of the inner values (\token{fst}
and \token{scd}), and thus one bit of its encoding is always used for the sign.
\smallskip

The \token{fst} value of the range literal is constructed before the
\token{scd} value. In the following example, the function \token{foo} is
called before the function \token{bar}.

\begin{lstlisting}[style=coloredverbatim]
fn foo ()-> i32 {
  println ("In foo.");
  12
}

fn bar ()-> i32 {
  println ("In bar.");
  1
}

let a = (foo ()) .. (bar ());
\end{lstlisting}

\subsection {Mutability and memory alignement}

As one might expect, range values do not share any data, and every field
included in the value is replicated during data movement. Therefore, there is no
concern for mutability, rendering the type not aliasable. A mutable range has
the ability to modify its internal fields, but even if the range type is a
compound type, it operates precisely like a scalar type since it can never
contain any borrowed data. In memory a range value has the same memory
alignement as the structure (cf. Chapter~\ref{chap:structures}) presented in the
following source code where \token{T} is the inner type of the range.

\begin{lstlisting}[style=coloredverbatim]
struct Range {T} {
  let fst : T;
  let scd : T;
  let step : T;
  let contain : bool;
}
\end{lstlisting}

\subsection {Properties}

Range type properties can be accessed through the operator \token{::} applied
to a type expression. The following properties include:

\begin{center}\begin{adjustbox}{max width=\linewidth}
  \begin{tabular}{|l|ll|}
    \hline
    Name & Meaning & Type\\
    \hline
    \hline
    \texttt{init} & The initial value ranging from \texttt{T::init} & \texttt{typeof (x)}\\
    & to \texttt{T::init} with a step of \texttt{T::init} and & \\
    & with \texttt{contain} set to \texttt{false} where \texttt{T} is & \\
    & the inner type (e.g. \texttt{i32} for \texttt{..i32}). &\\
    \hline
    \texttt{typeid} & A string encoding the name of the type & \texttt{[c8]} \\
    \hline
  \end{tabular}
\end{adjustbox}\end{center}


\subsection {Binary operators}

Binary operators are divided into 4 groups:

\begin{itemize}
  \setlength\itemsep{-4pt}
\item Access: The operator \token{.} is utilized to access the field of the
  range type. The right operand is the name of the field to access. The fields
  are listed in the table below.

  \begin{center}\begin{adjustbox}{max width=\linewidth}
    \begin{threeparttable}
      \begin{tabular}{|l|ll|}
        \hline
        Name & Value & Type\\
        \hline
        \hline
        \texttt {fst} & The first value of the range & \textit{T}$^{1^{\phantom{j}}}$ \\
        \texttt {scd} & The second value of the range & \textit{T}$^{1^{\phantom{j}}}$ \\
        \texttt {step} & The step of the range & \textit{S}$^{2^{\phantom{j}}}$ \\
        \texttt {contain} & The field describing wether or  & \texttt{bool} \\
        & not the scd value is contained in the range &\\
        \hline
      \end{tabular}
      \begin{tablenotes}
      \item[1.] \texttt{\_\_pragma!inner (typeof (x), 0)}
      \item[2.]\small \textit{Signed version of inner type for integer type
        ranges, an integer type for character type ranges, and
        \texttt{\_\_pragma!inner (typeof(x), 0)} for float type ranges.}
      \end{tablenotes}
    \end{threeparttable}
\end{adjustbox}\end{center}

Accessed fields are only mutable if the range is also mutable.

\item Contains: The \token{in} and \token{!in} operators verify the presence
  of a value in a range. The left operand must match the inner type of the right
  operand, and the right operand must be a range type.

\item Comparison: Ranges can be compared using the operators \token{==} and
  \token{!=}. This checks the equality or inequality of each field within the
  range. It is important that the left and right operands are of the same type.

\item Affectation: A range value can be an lvalue if and only if it is mutable.
  \begin{lstlisting}[style=coloredverbatim]
let mut a = 0 .. 7;

a = 7 .. 1;
  \end{lstlisting}

\end{itemize}

\subsection {Range iteration}

 Ranges are iterable types, so they can be used as the iterable value of a
 \token{for} loop. Only one immutable variable can be declared when iterating
 over a range value. This iterator variable takes the value of the \token{fst}
 field of the range and increments by the \token{step} field until it reaches
 the \token{scd} field. If the range is a containing range (i.e. the
 \token{contain} field is true), then the \token{scd} field is included in
 the iteration. Optimisations are performed if the inner values of the range are
 known at compile time.

\begin{lstlisting}[style=coloredverbatim]
for i in 0 .. 7 {
    print (i, ' '); // 0 1 2 3 4 5 6
}

for i in 0 ... 7 {
    print (i, ' '); // 0 1 2 3 4 5 6 7
}

for i in 7 .. 0 {
    print (i, ' '); // 7 6 5 4 3 2 1
}
\end{lstlisting}

For more information on \token{for} loops, see the Section~\ref{sec:for_loop}.

\vfill%
\pagebreak

\section*{Exercises}
\subsection*{Exercise 1}

Write a Ymir program that prints the following to the console output :
\begin{verbatim}
Hello!
I wrote my first Ymir program.
Ain't that fun?
\end{verbatim}

\subsection*{Exercise 2}
The following program contains several errors:

\begin{lstlisting}[style=bashVerb, caption=Ymir program with errors]
*/ Comments of the main function /*
fn main {
   println ("If this text")
   println ("appears on the screen, ');
   println ('you got it right').
)
\end{lstlisting}
Correct the errors and compile the program to test your corrections.

\subsection*{Exercise 3}

What does the following program output to the screen?
\begin{lstlisting}[style=coloredverbatim]
use std::io;

fn pause() {
   print ("BREAK");
}

fn main () {
   println ("\nDear reader, ");
   print ("have a ");
   pause();
   println ("!");
}
\end{lstlisting}

\vfill%
\pagebreak

\forceevenpage{}

\section*{Solutions}
\subsection*{Exercise 1}

\begin{lstlisting}[style=coloredverbatim, caption=Solution for exercise 1]
use std::io;

fn main () {
  println ("Hello!");
  println ("I wrote my first Ymir program.");
  println ("Ain't that fun?");
}
\end{lstlisting}

\subsection*{Exercise 2}

\begin{lstlisting}[style=coloredverbatimCorrect, escapechar=@]
@\hcb{\color{purple}{use} \color{black}{std::io;}}@

@\hcb{/*}@ @\textit{Comments of the main function}@ @\hcb{*/}@
fn main () {
   println (@\color{black!30!applegreen}{"If this text"}@)@\hcb{;}@
   println (@\color{black!30!applegreen}{"appears on the screen,}@ @\hcb{"}@);
   println (@\hcb{"}@@\color{black!30!applegreen}{you got it right}@@\hcb{"}@);
@\hcb{\}}@
\end{lstlisting}

\subsection*{Exercise 3}
The output starts with a line return.
\begin{lstlisting}[style=bashVerb]

Dear reader,
have a BREAK!
\end{lstlisting}

