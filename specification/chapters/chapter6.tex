\chapter{Variables and memory management}
\pagecolor{gray!10!white}
\label{chap:chap6}

A variable in Ymir, like in other programming languages, is a symbolic name that
represents a memory location where a value is stored. When the variable name is
used, the program accesses this memory location to retrieve or modify the value.
Depending on the type of variable that is written out, the memory access can be
made differently (e.g., reference variables) or with varying permissions (e.g.,
writing permissions). This chapter presents the different types of variables
that can be defined in Ymir, their memory representation, and their permissions.
All the concepts presented in this chapter are crucial for understanding how
memory is managed in Ymir.

\minitoc%

\section{Preamble}%
\label{sec:preamble_compound_types}

Before presenting the specifications of the different compound types, let's
define some useful language elements.

\begin{itemize}
\item \textit{Data borrowing} -- A data borrowing value is a value that points
  to data that is not automatically copied when the value is copied. The
  simplest example of data borrowing is the use of a pointer type, where the
  value of the pointer provides information about a memory address containing
  data. The data that the pointer points to is not copied when the pointer is
  copied. An illustration of this is given in figure~\ref{fig:(chap4):data_borrowing}.
  In this example, both variables \token{x} and \token{y} borrow the data
  located on the heap at address \token{0xfa45b987}. Copying the value of the
  variable \token{y} to another variable does not make a copy of the borrowed
  value (which is on the heap), but only copies the value directly inside
  \token{y}, which in this case is on the stack. The borrowed data can be on
  the stack or on the heap, and the same can be said for the borrowing data.
  Such memory is called ``borrowed'' because it can live longer than the value
  pointing to it, and the pointer only temporarily borrows its value for reading
  or writing.

  In the case of borrowing data, mutability is crucial because multiple variables
  refer to the same segment of memory. This can lead to unpredictable side
  effects. Memory mutability is used to manage such behaviour and to ensure that
  only a few variables have mutable access to values. Most importantly, it is a
  safeguard against the accidental creation of mutable memory references.


\end{itemize}

\begin{figure}[H]
  \centering
  \begin{adjustbox}{max size=\linewidth}
    \begin{tikzpicture}
      % ================ Axis ========================
      \draw (0, 0) -- coordinate (LEFT) (0, 0);
      \draw[-Stealth](0, 10) -- coordinate (YAx) (0, 1.5);
      \draw[-Stealth](7.2, 10) -- coordinate (YAx) (7.2, 1.5);
      \draw[Box] (3.5, 10.7) -- (3.5, 10.7) node[anchor=center]{\textbf{STACK}};

      \filldraw (0, 6.5) circle (1pt) node[align=center, left] {+0};
      \filldraw (0, 5.5) circle (1pt) node[align=center, left] {+8};
      \filldraw (0, 4.5) circle (1pt) node[align=center, left] {+16};
      \filldraw (0, 3.5) circle (1pt) node[align=center, left] {+24};
      \filldraw (0, 2.5) circle (1pt) node[align=center, left] {+32};

      %% % ================ Foo ========================
      \fill[black!10] (0.2, 10) rectangle (7, 8);
      \draw[Box] (3.5, 7.6) -- (3.5, 7.6) node[anchor=center]{\textbf{.}};
      \draw[Box] (3.5, 7.4) -- (3.5, 7.4) node[anchor=center]{\textbf{.}};
      \draw[Box] (3.5, 7.2) -- (3.5, 7.2) node[anchor=center]{\textbf{.}};

      \draw[line width=1pt] (0.18, 6.9) rectangle (7.02, 2.48);
      \fill[olive!10] (0.2, 6.5) rectangle (7, 2.5);
      \draw[Box] (3.5, 6.7) -- (3.5, 6.7) node[anchor=center]{\emph{foo}};

      \draw[Box] (0.5, 5.5) -- (0.5, 5.5) node[anchor=center]{\textbf{\textit{x}}};
      \draw[line width=0.005pt] (0.3, 6.4) rectangle (6.8, 4.6);
      \draw[line width=0.005pt] (0.7, 5.5) rectangle (6.8, 5.5);

      \draw[Box] (3.5, 6) -- (3.5, 6) node[anchor=center]{\emph{len~=}~$3$};
      \draw[Box] (3.5, 5) -- (3.5, 5) node[anchor=center]{\emph{ptr~=}~0xfa45b987};

      \draw[Box] (0.5, 3.5) -- (0.5, 3.5) node[anchor=center]{\textbf{\textit{y}}};
      \draw[line width=0.005pt] (0.3, 4.4) rectangle (6.8, 2.6);
      \draw[line width=0.005pt] (0.7, 3.5) rectangle (6.8, 3.5);

      \draw[Box] (3.5, 4) -- (3.5, 4) node[anchor=center]{\emph{len~=}~$3$};
      \draw[Box] (3.5, 3) -- (3.5, 3) node[anchor=center]{\emph{ptr~=}~0xfa45b987};


      %% % ================ Heap ========================

      \draw[-Stealth](10, 10) -- coordinate (YAx) (10, 1.5);
      \draw[-Stealth](17.2, 10) -- coordinate (YAx) (17.2, 1.5);
      \filldraw (17.2, 7.5) circle (1pt) node[align=center, right] {0xfa45b987};

      % ================ Data ========================
      \draw[Box] (13.5, 10.7) -- (13.5, 10.7) node[anchor=center]{\textbf{HEAP}};
      \fill[black!10] (10.2, 10) rectangle (17, 8);
      \fill[teal!20] (10.2, 7.9) rectangle (17, 5);
      \draw[line width=0.005pt] (10.4, 7) rectangle (16.8, 7);
      \draw[line width=0.005pt] (10.4, 6) rectangle (16.8, 6);

      \draw[Box] (13.5, 7.5)  (13.8, 7.5) node[anchor=center]{$1$};
      \draw[Box] (13.5, 6.5)  (13.8, 6.5) node[anchor=center]{$2$};
      \draw[Box] (13.5, 5.5)  (13.8, 5.5) node[anchor=center]{$3$};

      \draw[thick,-Stealth,shorten >=1pt] (7.3, 5) to [out=0,in=180] node[below left, yshift=3mm] {} (9.9, 7.5);

      \draw[thick,-Stealth,shorten >=1pt] (7.3, 3) to [out=0,in=180] node[below left, yshift=3mm] {} (9.9, 7.5);

    \end{tikzpicture}
  \end{adjustbox}
  \vspace{-30pt}%
  \caption{\label{fig:data_borrowing}Example of data borrowing}
\end{figure}
%
%% \vspace{2pt}%


\begin{itemize}
  \setlength\itemsep{-4pt}
\item \textit{Data movement} -- Data movement is the process of copying a value
  from one segment of memory to another. This can be done by allocation, memory
  copy (or deep copy), function calls and other operations. Data movements are
  always explicit, which ensures that no unwanted copies are made that could
  potentially slow down the program.

\item \textit{lvalue} -- An lvalue refers to the left operand in a \textit{data
  movement} operation. In other words, it is an expression that refers to a
  segment of memory that is modified as a result of the \textit{data movement}.
  An lvalue can take various forms, including a simple variable, a function
  parameter, an index operation, and more.

\item \textit{rvalue} -- An rvalue is the right operand involved in a
  \textit{data movement} operation. A rvalue can be aliased (using the keyword
  \token{alias}), referenced (keyword \token{ref}), copied with
  (\token{copy}, \token{dcopy}), or it can be implicit. Implicit means that
  the \textit{data movement} does not borrow mutable data and can therefore be
  allowed implicitly. In the case of an implicit \textit{data movement} there is
  no copying of borrowed data and there is no new borrowing of mutable memory.

\end{itemize}







\section{Pointers}%
\label{sec:pointer_type}

Pointers are values storing an address of memory. Pointer types are described
using the \token{*} token followed by a type (e.g. \token{*i32} describing a
pointer to a \token{i32} value). In the beta version of Ymir (compiler written
in c++) the operator \token{\&} was used, it was changed as it was also used to
refer to object instances that are in a way pointers but have very different
behavior.

\subsection {Literals}

The keyword \token{null} is used to describe a pointer pointing to nowhere.
This is the only literal that can be used as a pointer value.

\subsection {Construction}

To construct a pointer, the unary operator \token{\&} can be used on a lvalue
(for example, a variable). This operator returns the address of the value
referenced by the operand, i.e. \token{\&a} returns the address of the segment
of memory referenced by the variable \token{a}. For the sake of simplicity, we
can say that we are retrieving the address of the variable \token{a}.

\begin{lstlisting}[style=coloredverbatim]
let a = 12;
let b : *i32 = &a;
\end{lstlisting}

\subsection {Mutability}

Because pointers borrow data from another value (value pointed by the pointer),
their mutability is important. A pointer has two level of mutability:
\begin{enumerate}
  \setlength\itemsep{-4pt}
\item \token{mut *T}, In this case, the pointer can be changed, but the value
  inside the pointer cannot be changed.
\item \token{mut *(mut T)}, In this case, both the value to which the pointer
  is pointing and the pointer itself can be changed.
\end{enumerate}

A mutable pointer (\textit{level 1}) means that if the pointer is contained
inside another compound type or variable, then the value it points to can be
changed. Checking for mutability is done at compile time when borrowing a value
to construct a pointer value.
\smallskip

\begin{lstlisting}[style=coloredverbatim]
let dmut a : *i32 = null;

let b = 12;
a = &b; // not allowed 'b' is not mutable
*a = 24; // but it would be modified by this operation


let mut c = 11;
a = &c; // allowed c is mutable
*a = 24; // modify the value of c is allowed
\end{lstlisting}

\smallskip
The keyword \token{alias} must be used on the right operand when data borrowing
is transferred to the left operand. In practice, this means that if the
mutability of the left operand is second level (i.e. \token{mut *(mut T)}), the
keyword \token{alias} must be used, and the right operand must also be second
level mutable. The keyword can be omitted if the aliasing is obvious (i.e. by
function return, or construction such as the unary operator \token{\&}).

\subsection {Properties}

Pointer type properties can be accessed using the \token{::} operator on a type
expression. The properties are as follows: following:

\begin{center}\begin{adjustbox}{max width=\linewidth}
  \begin{tabular}{|l|ll|}
    \hline
    Name & Meaning & Type\\
    \hline
    \hline
    \texttt{init} & The initial value \texttt{null} & \texttt{typeof(x)}\\
    \hline
    \texttt{typeid} & A string encoding the name of the type & \texttt{[c8]} \\
    \hline
  \end{tabular}
\end{adjustbox}\end{center}

\subsection {Casting}

A pointer type can be cast to any pointer type using the cast operator
\token{cast!T(V)}. A pointer is a really low-level type with few guarantees,
but some operations rely on this possibility to perform generic operations
(common trait \token{Packable} for example). The mutability of the result is
the same as the mutability of the operand, if the value is explicitely aliased,
it is immutable otherwise. This is the only allowed cast on pointer types.
\smallskip

\begin{lstlisting}[style=coloredverbatim]
  let dmut a : [i32]  = copy [12];
  let dmut b : *[i32] = &a;

  // c : dmut *[f32];
  let dmut c = cast!{*[f32]} (alias b);

  // not allowed, from *[f32] to dmut *[f32];
  let dmut d = cast!{*[f32]} (b);
\end{lstlisting}

\subsection {Unary operators}

The unary operator \token{*} is used on a pointer value to dereference it and
access the value pointed to by the pointer. This operation is unsafe and may
cause the program to crash. If the operation does not crash, it does not
necessarily mean that the pointer was created correctly. This operation is
unsafe and can only be used in a \token{unsafe} context.
\smallskip

\begin{lstlisting}[style=coloredverbatim]
let i = 10;
let x = &i;

unsafe {
  let j = *x;
}
\end{lstlisting}

\subsection {Binary operators}

Binary operators are divided into 3 groups:
\begin{itemize}
  \setlength\itemsep{-4pt}
\item Arithmetic: Pointer arithmetic is allowed with a \token{usize} as the
  right operand. Unlike the C language, the arithmetic does not depend on the
  size of the data pointed to by the pointer. The operation adds a number of
  bytes to the address, which means that the addition operation with a left
  operand whose value is \token{0xabc0} and a right operand \token{8us} will
  always have the value \token{0xabc8}, regardless of the type of content
  pointed to by the pointer. The behaviour is not the same with the index
  operator. The type of the result of the operation is always the same as the
  type of the left operand.

  \begin{center}\begin{adjustbox}{max width=1\linewidth}
    \begin{tabular}{|c|lll|}
      \hline
      Operator & Operation & Commutative & Example \\
      \hline
      \hline
      \texttt{+} & Addition & Yes & \texttt{\&a + 2us} \\
      \texttt{-} & Substraction & No & \texttt{\&a - 2us} \\
      \hline
    \end{tabular}
  \end{adjustbox}\end{center}

\item Logical: Comparison operators always return a value of type \token{bool}
  and can only be used if the two operands are of the same pointer type (e.g.
  \token{*i32} with \token{*i32}).

  \begin{center}\begin{adjustbox}{max width=1\linewidth}
      \begin{threeparttable}
        \begin{tabular}{|c|lll|}
          \hline
          Operator & Operation & Commutative & Example \\
          \hline
          \hline
          \texttt{==} & Equality test & Yes & \texttt{(\&a == \&a) == true}\\
          \texttt{!=} & Inequality test & Yes & \texttt{(\&a != \&a) == false}\\
          \texttt{<} & Lower than & No & \texttt{(\&a < \&a + 1us) == false}\\
          \texttt{>} & Greater than & No & \texttt{(\&a > \&a - 1us) == true}\\
          \texttt{>=} & Greater or equal$^{1^{\phantom{j}}}$ & No & \texttt{(\&a >= \&a - 1us) == true}\\
          \texttt{<=} & Lower or equal$^{1^{\phantom{j}}}$ & No & \texttt{(\&a <= \&a - 1us) == false}\\
          \hline
        \end{tabular}
    \end{threeparttable}
    \end{adjustbox}\end{center}

\item Affectation: The affectation operators are usable when the two operands
  are of exactly the same pointer type. The mutability level of the left operand
  must be less than or equal to the mutability level of the right operand.
  Affection operators can be mixed with arithmetic operators (e.g. \token{+=},
  \token{-=}). In this case the operation is rewritten as \token{x = x + y}
  and \token{y} must be a value of type \token{usize}.

  \begin{lstlisting}[style=coloredverbatim]
let mut a = 11;
let dmut b = &a;

let mut c = &a;
b = c; // not allowed it will discard the const property
c = b; // No problem the mutability level of c is lower than the one of b

c += 1us;

let dmut d = &a;
b = alias d; // alias is needed, data is borrowed
  \end{lstlisting}

\end{itemize}

\subsection {Index operator}

The index operator can be used on a pointer left operand with an int value as
the index right operand. The result of the operation is the dereferencing of the
pointer value by the offset of the value used as index. Unlike pointer
arithmetic using the \token{+} and \token{-} operators, the index operator
takes into account the size of the data pointed to by the pointer, i.e. the
index operation \token{(\&a)[7]} is strictly transformed into \token{*(\&a +
  (7us * sizeof (\_\_pragma!inner (typeof(\&a), 0)))}. This operation is unsafe
because it dereferences a raw pointer.

\smallskip

\begin{lstlisting}[style=coloredverbatim]
let mut a = 12;
let dmut b = &a;

unsafe {
  b [0] = 89;
}
assert (a == 89);
\end{lstlisting}

\smallskip

The mutability of the result value depends on the mutability level of the
pointer operand. If the mutability level of the pointer operand is 2, then the
result can be used as lvalue. The above example produces the following Ymir Intermediate Language code.

\begin{lstlisting}[style=myilVerb]
frame : main::main ()-> void {
    let mut a : mut i32 = 12;
    let mut b : mut *(mut i32) = &(a);
    {
        *((b + (cast!{usize}(0) * 4us))) = 89;
        <unit-value>
    };
    core::exception::abort ((a == 89), (""s8)[]);
    <unit-value>
}
\end{lstlisting}

\vfill%
\pagebreak

\section{Characters}
\label{sec:char_type}

Character types are used to encode characters (ASCII or Unicode). There are
three character types \token{c8}, \token{c16} and \token{c32} with a size of
\token{8}, \token{16} and \token{32} bits respectively. These char types are
encoding values in utf-8, utf-16 and utf-32.

\begin{center}
  \begin{tabular}{ll}
    Value & Content\\[0pt]
    \hline
    \texttt{\textbackslash{}a} & Alert beep, (Bell)\\[0pt]
    \texttt{\textbackslash{}b} & Backspace\\[0pt]
    \texttt{\textbackslash{}f} & Page break\\[0pt]
    \texttt{\textbackslash{}n} & New line\\[0pt]
    \texttt{\textbackslash{}r} & Carriage return\\[0pt]
    \texttt{\textbackslash{}t} & Horizontal tab\\[0pt]
    \texttt{\textbackslash{}v} & Vertical tab\\[0pt]
    \texttt{\textbackslash{}\textbackslash{}} & Backslash\\[0pt]
    \texttt{\textbackslash{}'} & Apostrophe\\[0pt]
    \texttt{\textbackslash{}"} & Double quotation mark\\[0pt]
    \texttt{\textbackslash{}u\{\}} & Unicode\\[0pt]
  \end{tabular}
  \captionof{table}{\label{tab:escape_chars} Escape characters}
\end{center}

\subsection{Properties}
\label{sec:orgf9fbc31}

The properties of char types can be accessed by using the \token{::} operator
on a type expression. The properties are as follows:

\begin{center}
  \begin{adjustbox}{max width=\linewidth}
    \begin{tabular}{|l|ll|}
      \hline
      Name & Meaning & Type\\[0pt]
      \hline
      \hline
      \texttt{init} & The initial value \texttt{\textbackslash{}u\{0\}} & \texttt{typeof(x)}\\[0pt]
      \hline
      \texttt{typeid} & A string encoding the name of the type & \texttt{[c8]}\\[0pt]
      \hline
    \end{tabular}
  \end{adjustbox}
\end{center}


\subsection{Literals}
\label{sec:org73c4919}

Char literals are enclosed by the token \token{'} and can be described using three forms:

\begin{enumerate}
  \setlength\itemsep{-4pt}
\item Using the direct representation of the character (e.g. \tokennolst{π}),
\item Using an escape character. The escape chars are described in the
  table~\ref{tab:escape_chars}.

\item Using an int literal representation of Unicode. To avoid confusing the int
  literal representation with the literal of the int itself, the int literal
  must be encoded using the escape character \tokennolst{\textbackslash{}u} and the
  tokens \token{\{} and \token{\}}. For example
  \tokennolst{\textbackslash{}u\{0x263A\}}, \tokennolst{\textbackslash{}u\{0b1101\}} or
  \tokennolst{\textbackslash{}u\{10\}}.

\end{enumerate}

As with float or int literals, a suffix must be added at the end of the literal
to define the value with the correct type. For example, to define a \token{c32}
value containing the character \token{a}, write \token{'a'c32}. Literal values
without a suffix are considered to be of type \token{c8}.

\begin{lstlisting}[style=coloredverbatim]
let a : c32 = 'r'c32;
let b : c8 = '\u{10}';
let d = 'π'c32;
let e = '\n'c8;

assert (e == b);
\end{lstlisting}

\subsection{Casting}
\label{sec:org16d703f}

Char types can be cast using the cast operator. It is not possible to implicitly convert a char value to a value of another type.

\begin{itemize}
  \setlength\itemsep{-4pt}
\item To other char types: The cast operator can be used to convert a char of
  one size to a char of another size. This does not guarantee encoding validity.
  The standard library defines more complex transformations that respect
  encoding in the \token{std::conv} module.

\item To integer types: The cast operator can be used to transform a char value
  of type \token{c8} into a \token{u8}, a \token{c16} into a \token{u16} and
  a \token{c32} into a \token{u32}. The transformation does not change the
  value in any way (exactly the same bits before and after the cast).

\end{itemize}


\subsubsection{Implicit casting}

The implicit casting of char values known at \token{cte} follows the same
rules as described for integer \token{cte} values.

\begin{lstlisting}[style=coloredverbatim, escapechar=@]
fn foo (f : c32) {}
fn foo (f : c8) {}

let a : c8 = 'a'c32; // ok, c32 is converted implicitely to c8

foo ('a'); // ok, using c8 definition
@\hb{foo ('a'c16)}@; // error, both definitions work
\end{lstlisting}

\subsection{Unary operators}
\label{sec:org78546fb}

No unary operators can be used on char types.

\subsection{Binary operators}
\label{sec:orge863f7d}

Binary operators on character types are divided into four groups.

\begin{itemize}
  \setlength\itemsep{-4pt}
\item Arithmetic: Binary arithmetic operators can be used with a char value and
  an unsigned int value (of the same size, e.g. for \token{c8} and
  \token{u8}). The result always takes the type of the char operand. It is
  impossible to add or subtract two char values, even if they are of exactly the
  same type.

  \begin{center}
    \begin{adjustbox}{max width=1.0\linewidth}
      \begin{tabular}{|c|lll|}
        \hline
        Operator & Operation & Commutative & Example\\[0pt]
        \hline
        \hline
        \texttt{+} & Addition & Yes & \texttt{'a' + 16u8 == 'q'}\\[0pt]
        \texttt{-} & Subtraction & No & \texttt{'q' - 16u8 == 'a'}\\[0pt]
        \hline
      \end{tabular}
  \end{adjustbox}\end{center}


  Char values can be used as right operands in arithmetic operations. The type of
  the result operation would still be the type of the char operand, and the int
  operand would still have to be of the same size as the char operand, so
  \token{('q' + 12u8) == (12u8 + 'q')}.

\item Logical: Binary logical operators can be used with two char values of
  exactly the same type. The result of the operation is always of type
  \token{bool}.

  \begin{center}
    \begin{adjustbox}{max width=1.0\linewidth}
      \begin{tabular}{|c|lll|}
        \hline
        Operator & Operation & Commutative & Example\\[0pt]
        \hline
        \hline
        \texttt{>} & Greater than & No & \texttt{('q' > 'a') == true}\\[0pt]
        \texttt{<} & Lower than & No & \texttt{('q' < 'a') == false}\\[0pt]
        \texttt{>=} & Greater or equal & No & \texttt{('q' >= 'q') == true}\\[0pt]
        \texttt{<=} & Lower or equal & No & \texttt{('b' <= 'r') == true}\\[0pt]
        \texttt{==} & Equal & Yes & \texttt{('a' == 'a') == true}\\[0pt]
        \texttt{!=} & Not equal & Yes & \texttt{('a' != 'a') == false}\\[0pt]
        \hline
      \end{tabular}
  \end{adjustbox}\end{center}

\item Affectation: The affectation operator \token{=} is usable when the left
  operand is a mutable lvalue and the right operand is strictly the same char
  type as the left operand.

  The affectation operator can be mixed with an arithmetic operator \token{+=}
  and \token{-=}, in which case the right operand must be a value whose type is
  an unsigned int with a size exactly equal to the size of the char type of the
  left operand. The affectation \token{x += y} is rewritten as \token{x = x +
    (y)}, where the y operand always has a higher priority than the affectation
  operator.

  \begin{lstlisting}[style=coloredverbatim]
let mut a = 'a';

let b = a + 21u8;

a = 'e';
a += 7u8;

assert (b == 'v');
assert (a == 'l')
  \end{lstlisting}

\item Range: The range operator can be used on two char values whose types are
  strictly identical, creating a range value.

  \begin{center}
    \begin{adjustbox}{max width=1.0\linewidth}
      \begin{tabular}{|l|lll|}
        \hline
        Operator & Operation & Example & Interval\\[0pt]
        \hline
        \hline
        \texttt{..} & Range operator not inclusive & \texttt{'a' .. 'z'} & \texttt{[a;z[}\\[0pt]
            \texttt{...} & Range operator inclusive & \texttt{'a' ... 'r'} & \texttt{[a;r]}\\[0pt]
            \hline
      \end{tabular}
  \end{adjustbox}\end{center}


  The result value has a default increment of \token{1} and its inner type is
  the type of the operands. It can be increasing or decreasing depending on the
  values used to construct it.

\end{itemize}

\subsection{Overflowing}
\label{sec:orga9c18c5}

Compile time overflow checking is done on \textit{cte} values. The check ensures
that the chosen type is large enough to encode the value. There is no way to
check for overflow at runtime, and it can happen. It is also possible, due to
the encoding, that a value is not a valid unicode or ascii value if it is
created at runtime (e.g. \token{'π' + 501u32}).

\vfill%
\pagebreak

\section{Global name aliases}
\label{sec:global_alias_names}

A name alias can be used to simplify code by creating a shorthand reference to a
frequently used type or value. This can make the code more readable and easier
to maintain. The keyword \token{def} is used to create a name alias.

\subsection {Using name alias for types}

A name alias can be associated to a type, by using the keyword \token{def}
followed by an identifier, the token \token{:} and finally a type description.

\begin{lstlisting}[style=coloredverbatim]
def String : [c8];

let a : String = "Hello";
\end{lstlisting}

The alias is a global symbol and has access to other symbols as described in
Section~\ref{sec:symbol_protection}. Therefore, they can be used to export
private symbols within a module to external modules under certain conditions. In
the next example, the class \token{A} takes a template parameter that could be
associated with any value, but we only want to export the version that takes the
value \token{"extern"} as a template parameter without defining the class twice.
The alias declaration \token{X} at line 24 comes in handy, allowing this
exportation without making the symbol \token{A} accessible from outside modules.

\begin{lstlisting}[style=coloredverbatim]
use std::io;

mod inner {
    class A {value : [c8]} {
        pub self () {}

        cte if (value == "local") {
            pub fn local (self) {
                println ("Method for local use only");
            }
        } else {
            pub fn external (self) {
                println ("Method for external use");
            }
        }
    }

    pub fn foo () {
        let x = A!{"local"}::new ();
        x.local ();
    }

    // Exporting a specific specialization of A
    pub def X : A!{"extern"};
}


fn main () {
    let x = inner::X::new ();
    x.external ();

    inner::foo ();
}
\end{lstlisting}

This code demonstrates how to use name alias to create flexible and reusable
components while keeping certain implementation details protected within a
module. The above code is just an example, and there might be many other reasons
that justify the use of name aliases, such as code simplification,
maintainability, and abstractions, among others.

\subsubsection*{Type mutability}

The type associated with an alias is always immutable, with its mutability
effectively defined when the alias is used rather than when it is declared. For
example, in the following code, the alias \token{IntArray} is associated with a
\token{[i32]} slice type, and one cannot declare its mutability within the alias
declaration itself. At line 3, when the alias \token{IntArray} is used to
declare a variable, the use of the keyword \token{dmut} makes the type of the
variable deeply mutable. On the other hand, when the keyword \token{dmut} is not
used at line 5, the resulting type is \token{mut [i32]} instead of \token{mut
  [mut i32]}.

\begin{lstlisting}[style=coloredverbatim, escapechar=@]
def IntArray : @\hb{dmut}@ [i32]; // error, mutability change is not allowed here

let dmut a : IntArray = copy [1, 2, 3]; // inner values are mutable, due to dmut
let mut b : IntArray = a; // inner values are not mutable, no need to alias a

b [0] @\hb{=}@ 9; // mutability of IntArray is const by default
\end{lstlisting}

The reason for not allowing mutability change directly from the alias
declaration is to avoid hidden mutability changes. If the keywords \token{mut}
and \token{dmut} were not visible at the location of the variable declaration,
it would create a situation where the variable has mutable access to its data
without clear indication. This would necessitate checking the alias declaration
to ensure such hidden behavior does not occur. This approach contradicts the
philosophy of the Ymir language, which aims to place all behavior explanations
at the location where they effectively have an impact, thereby avoiding the
dispersion of information across the code.

\subsection{Using name alias for values}

A value can be associated to an alias name using a syntax close to the one used
to declare a type name alias, but by replacing the token \token{:} by the token
\token{=}.

\begin{lstlisting}[style=coloredverbatim]
def SuccessMessage = "Success !";
def ErrorMessage = "Failure..";

let test : bool = // ...;
if test {
  println (SuccessMessage);
} else {
  println (ErrorMessage);
}
\end{lstlisting}

As a name alias for a type, a name alias to a value is a global symbol, and thus
can be used to export private symbol to external modules under certain
conditions. In the next example, the function \token{foo} is not accessible fro
the the \token{main} function while the global alias \token{F} is. For that
reason at line 18 and 19, the function \token{foo} can be called using the alias
\token{F} that export an alias to the function prototype symbols. In this
example, the name alias actually exports an alias to multiple symbols - the two
functions \token{foo} declared within the module \token{inner}. It is during the
call expressions at line 18 and 19 that the a specialization is performed (at
compile time) to decide which of the two symbols is used.

\begin{lstlisting}[style=coloredverbatim]
use std::io;

mod inner {

  fn foo (i : i32) {
    println ("In first foo : ", i);
  }

  fn foo (i : i32, j : i32) {
    println ("In second foo : ", i + j);
  }

  pub def F = inner::foo; // with inner:: to make sure no other foo functions are aliased
}

fn main () {
  inner::F (12);
  inner::F (12, 23);
}
\end{lstlisting}

\subsubsection*{Value alias construction}

A name alias exists only during compile time, and therefore has no memory
location at runtime. It is not a constant global variable. For that reason, the
value attached to a name alias is constructed each time it is referenced. In the
next example, the name alias \token{F} is attached to a value that calls the
function \token{foo}. This alias is referenced twice at line 9 and 10, therefore
the function \token{foo} is called twice.

\begin{lstlisting}[style=coloredverbatim]
fn foo ()-> i32 {
  println ("Calling foo");
  12
}

def F = foo ();

fn main () {
  let _ = F;
  let _ = F;
}
\end{lstlisting}


This behavior has other implications, firstly because name aliases have no
memory location and are simply shorthands for constructing values, they cannot
be changed, and have no memory address.

\begin{lstlisting}[style=coloredverbatim, escapechar=@]
def V = 12;
def A = copy [1, 2, 3];

V @\hb{=}@ 89; // error, V is not a lvalue
A [0] @\hb{=}@ 8; // error, A is not an lvalue either
\end{lstlisting}

Secondly, when it involves value allocations and memory borrowing. In the next
example, the variable \token{a} and \token{b} borrows two different slices, that
are constructed when referencing the name alias \token{A}.

\begin{lstlisting}[style=coloredverbatim]
def A = copy [1, 2, 3];

let dmut a = A;
let dmut b = A;

a [0] = 9;
assert (b [0] == 1);
\end{lstlisting}

One could arguee that in this example, from the perspective of the variables
\token{a} and \token{b} an hidden allocation is performed. It was chosen to
allow this kind of behavior as the same observation can be made when calling a
function that returns an allocated value without needing the use of an
\token{alias} nor a new \token{copy}. From that understanding, a value name
alias can be seen as a function call that takes no parameter and that is inlined
at compile time.

\begin{lstlisting}[style=coloredverbatim]
fn foo ()-> dmut [i32];
def A = copy [1, 2, 3];

let dmut a = foo ();
let dmut b = A;
\end{lstlisting}

\vfill%
\pagebreak

\section{Global variable}%
\label{sec:global_variables}

A global variable is declared as a global symbol (within a module) using the
keyword \token{lazy} followed by an identifier. Global variables are thoroughly
presented in Chapter~\ref{chap:variables}.

\begin{lstlisting}[style=coloredverbatim]
lazy A = 12;

// Global variable protection system is the same as any global symbol
pub lazy B = A;
\end{lstlisting}

Global variables in Ymir are inherently lazy, meaning they are instantiated and
constructed only upon their first reference at runtime. This design addresses
the complexity arising from global variables referencing each other across
different modules. Constructing a dependency graph of symbols to ensure that
dependent global variables are created in the correct order would require a
sophisticated system. Such a system would be extremely complex to build and
sometimes impossible, as cycles could exist without being visible at compile
time (e.g., global variables from external packages). Hence, the lazy
instantiation approach simplifies this process. Laziness ensures that when a
global variable references another global variable, the referenced variable is
necessarily constructed.

\begin{lstlisting}[style=coloredverbatim]
lazy A = foo (); lazy B = bar (A);

fn foo ()-> i32 {
  println ("In foo");
  42
}

fn bar (a : i32)-> i32 {
  println ("In bar")
  a * 2
}

fn main () {
  println ("In main");
  println ("B = ", B); // first reference to B in the program
  println ("A = ", A); // second reference to A in the program
}
\end{lstlisting}

In the above listing, the construction of the global variable \token{B} depends
on the value of the global variable \token{A}. In other C-like languages (C++,
D, etc.), this would be problematic as there would be no guarantee that
\token{A} would be constructed before \token{B}. Thanks to the lazy system, when
the variable \token{B} is constructed, it triggers the construction of the
variable \token{A}. Therefore, \token{A} always has a value when \token{B} is
being constructed. The result of the above source code is presented in the next
listing.

\begin{lstlisting}[style=bashVerb, escapechar=@+]
In main
In foo
In bar
B = 84
A = 42
\end{lstlisting}

\vfill%
\pagebreak

\section{Character types}

\section{Scope guards}%
\label{sec:scope_guards}

Scope guards are a programming construct utilized to ensure that specific
operations are executed automatically upon exiting a scope. These guards are
commonly employed for tasks such as cleanup actions, error recovery, or managing
resources. There are two main types of scope guards: scope exiting and error
handling. The former, discussed in Section~\ref{sec:exit_guards}, is declared
following a scope declaration using one of three keywords: \token{exit},
\token{success}, or \token{failure} (e.g., \token{\{ ... \} exit \{ ... \}}). The
latter, discussed in Section~\ref{sec:catching_errors}, is declared with the
keyword catch and is utilized to handle exceptions thrown within a specific
scope, employing the syntax \token{\{...\} catch \{ Pattern => V \}}


\begin{lstlisting}[style=coloredverbatim]
fn foo ()-> i32
  throws AssertError;

{
  foo ();
} exit {
  println ("Called foo");
} catch {
  AssertError () => {
    println ("Foo has failed!");
  }
}
\end{lstlisting}

\subsection{Exit guards}
\label{sec:exit_guards}

The \token{exit} guard serves as a scope guard designed to execute an action
when a scope is exited, regardless of the success or failure of the guarded
scope. In contrast, the scope guards \token{failure} and \token{success} are
activated under specific conditions upon exiting the scope. The guard
\token{failure} is triggered when the scope throws an exception, while the guard
\token{success} is triggered when the scope exits normally. These scope guards
can be highly beneficial for managing resources that require disposal or for
calling functions that must execute at the end of a scope.

\smallskip

In the provided example, the code at line 10 is executed irrespective of whether
the \token{foo} function succeeds or fails. Conversely, the code at line 14 is
executed solely if the \token{foo} function exits normally, whereas the code at
line 12 is exclusively executed in the event of \token{foo} failing.

\begin{lstlisting}[style=coloredverbatim]
fn foo ()-> i32
  throws AssertError
{
  // ...
}

{
  foo ();
} exit {
  println ("Foo was exited");
} failure {
  println ("Foo failed");
} success {
  println ("Foo succeeded");
}
\end{lstlisting}

Exit scope guards lack access to variables declared within the scope they guard.
Specifically, \token{exit} and \token{failure} cannot ensure that variables
declared within the guarded scope are constructed. While \token{success} could
potentially guarantee this access, it adheres to the same behavior as other
scope guards and also lacks access to variables declared within the guarded
scope. In the following example, line 4 is prohibited as the variable \token{a}
no longer exists (and might have never been constructed).

\begin{lstlisting}[style=coloredverbatim, escapechar=@]
{
  let a = foo ();
} exit {
  println ("Foo returned : ", @\hb{a}@); // error, a does not exists here
}
\end{lstlisting}

Exit scope guards cannot throw exceptions, return early from functions, break
loops, or generate values. They may execute in contexts where the program is
already throwing an exception (e.g., the \token{failure} scope is only triggered
in such contexts) or in contexts that have already triggered a break or an early
function return. Allowing exit scope guards to throw exceptions, return early,
or break loops would result in two exceptions being thrown simultaneously or two
values being returned, which is nonsensical. Additionally, scope guards do not
generate values; the value of a guarded scope is the value generated by the
scope itself. Scope guards are solely used to execute actions once this value is
generated (or failed to be generated), not to handle errors or provide a
different value. For error handling, a \token{catch} scope guard can be
utilized, as will be presented in the subsequent subsection.

\begin{lstlisting}[style=coloredverbatim, escapechar=@]
let a = {
    1
} exit {
  @\hb{2}@ // error, scope guard value must be of type /void/
};
// If it was allowed, what would be the value of /a/ here?

{
  throw AssertError ("First exceptions");
} failure {
  @\hb{throw AssertError ("Second exception");}@ // error, scope guard can't throw exceptions
};
// If it was allowed, which would be the thrown error?

{
  return 1;
} success {
  @\hb{return 8}@; // error, exit scope guard cannot return a value
}
// And what would the function actually return ?
\end{lstlisting}

Exit scope guards can indeed be utilized to dispose of a class object, as
demonstrated in the following example. It's important to note that defining the
variable \token{f} within the scope guarded by the exit at line 6 wouldn't be
feasible, as it wouldn't be accessible within the scope guard and thus wouldn't
be disposable. However, it is not the preferred method. Instead, a disposing
scope declaration (refer to Section~\ref{sec:dispose_block}) is recommended for
such purposes.

\begin{lstlisting}[style=coloredverbatim]
use std::fs; // for File

let dmut f = File::create ("file.txt", write-> true);
{
  f.write ("content");
} exit {
  f.dispose ();
}

f.write ("other content"); // ok, but file was disposed so it will throw
\end{lstlisting}

For code clarity, the guards \token{failure} and \token{success} should only be
used if the guarded scope can throw exceptions. Otherwise, they would either
never be executed or be strictly equivalent to an \token{exit} guard.

\begin{lstlisting}[style=coloredverbatim, escapechar=@]
{
  println ("In scope");
} @\hb{failure}@ { // error, never used
  println ("Failed to print ??");
} @\hb{success}@ { // error, always executed, thus exit must be used instead
  println ("Succeed to print");
}
\end{lstlisting}

\vfill%
\pagebreak

