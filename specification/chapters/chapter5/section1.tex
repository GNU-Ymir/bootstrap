\section{Modules}%
\label{sec:modules}

A module is a first-level symbol that can contain other types of symbols. Any
other type of symbol is necessarily contained within a module.

\subsection{File module}

A source file declares a module symbol. Its name can be specified explicitly
with the keyword \token{in} as the first line of code. The purpose of this line
is for documentation purposes only, any comment above this line is considered
part of the documentation of the module symbol. The documentation syntax is
described in Chapter~\ref{chap:documentation}. The name of the module must
be the same as the file it is described in. For example, the file
\textit{foo.yr} should start with the line \token{in foo;}. This line is
optional.


\begin{lstlisting}[style=coloredverbatim]
/**
 * Documentation of the /foo/ module.
 * @Authors: John Doe
 * @Licence: GPLv3
 */
in foo;
\end{lstlisting}

A module can enclose other modules with the keyword \token{mod}. The name of
the enclosed module follows the keyword, and must be the name of the file being
enclosed. It can be a sibling file, or a child file (in the subdirectory with
the same name as the parent module). If both files exist (child and sibling),
the compiler will generate an error. Depending on whether they are child files
(in a directory whose name is the name of the enclosing module) or sibling
files, enclosed modules are either child or sibling. For example, assuming the
file hierarchy shown in Figure~\ref{fig:(chap5):file_hierarchy}, the module located in
the file \token{./foo/bar.yr} and enclosed by the module \token{./foo.yr} will
be named \token{foo::bar} whereas the module located in \token{./baz.yr}, even
if enclosed by the module \token{foo}, will be named \token{baz}. A complete
example is given below for the file hierarchy shown in
figure~\ref{fig:(chap5):file_hierarchy}.

%% \tikzstyle{every picture}+=[remember picture]
%% \tikzstyle{mProjectPP}=[circle,fill=yellow!70, minimum size=20pt,inner sep=0pt]
%% \tikzstyle{label}=[rectangle, fill=white, minimum size=40pt, inner sep=0pt]
%% \tikzstyle{Box}=[rectangle, fill=white, minimum size=40pt, inner sep=0pt]
%% \tikzstyle{Box2}=[rectangle, fill=white, minimum size=20pt, inner sep=0pt]
%% \definecolor{applegreen}{rgb}{0.55, 0.71, 0.0}
%% \definecolor{antiquebrass}{rgb}{0.8, 0.58, 0.46}

\begin{center}
  \begin{tikzpicture}
    % ================ Axis ========================

    \draw (0, 0) -- coordinate (LEFT) (0, 0);

    \draw[Box] ($(LEFT.east)+(0, 0)$) -- ($(LEFT.east)+(0, 0)$) node[anchor=center]{.};
    \draw[Box] ($(LEFT.east)+(1.5, -0.9)$) -- ($(LEFT.east)+(1.5, -0.9)$) node[anchor=center]{\color{blue} {\textbf{\textit{foo/}}}};
    \draw[line width=1pt] ($(LEFT.east)+(0,-0.5)$) -- ($(LEFT.east)+(0,-0.9)$) -- ($(LEFT.east)+(1,-0.9)$);

    \draw[Box] ($(LEFT.east)+(1.7, -0.5)$) -- ($(LEFT.east)+(1.7, -0.5)$) node[anchor=center]{\textbf{\textit{baz.yr}}};
    \draw[line width=1pt] ($(LEFT.east)+(0,-0.2)$) -- ($(LEFT.east)+(0,-0.5)$) -- ($(LEFT.east)+(1,-0.5)$);

    \draw[line width=1pt] ($(LEFT.east)+(1.25,-1.2)$) -- ($(LEFT.east)+(1.25,-1.3)$) -- ($(LEFT.east)+(2.3,-1.3)$);
    \draw[Box] ($(LEFT.east)+(3, -1.3)$) -- ($(LEFT.east)+(3, -1.3)$) node[anchor=center]{\textbf{\textit{bar.yr}}};

    \draw[line width=1pt] ($(LEFT.east)+(1.25,-1.2)$) -- ($(LEFT.east)+(1.25,-1.8)$) -- ($(LEFT.east)+(2.3,-1.8)$);
    \draw[Box] ($(LEFT.east)+(3, -1.8)$) -- ($(LEFT.east)+(3, -1.8)$) node[anchor=center]{\textbf{\textit{qux.yr}}};


    \draw[line width=1pt] ($(LEFT.east)+(0,-0.5)$) -- ($(LEFT.east)+(0,-2.2)$) -- ($(LEFT.east)+(1,-2.2)$);
    \draw[Box] ($(LEFT.east)+(1.7, -2.2)$) -- ($(LEFT.east)+(1.7, -2.2)$) node[anchor=center]{\textbf{\textit{foo.yr}}};

  \end{tikzpicture}
\end{center}
\captionof{figure}{\label{fig:file_hierarchy} Example of file hierarchy}

% \end{figure}


\noindent 1) Root directory :

\vspace{-5pt}%
\begin{minipage}[t][][t]{0.47\linewidth}
\begin{lstlisting}[caption=\textit{./foo.yr}, style=coloredverbatim]
in foo;

mod baz;
pub mod bar;
mod qux;

\end{lstlisting}
\end{minipage}\hspace{5pt}%
\begin{minipage}[t][][t]{0.47\linewidth}
\begin{lstlisting}[caption=\textit{./baz.yr}, style=coloredverbatim]
in baz;

pub fn pubFuncInBaz () {}
fn prvFuncInBaz () {}
\end{lstlisting}
\end{minipage}


\noindent 2) Subdirectory \textit{foo/}:

\vspace{-5pt}%
\begin{minipage}[t][][t]{0.47\linewidth}
\begin{lstlisting}[caption=\textit{./foo/bar.yr}, style=coloredverbatim]
in bar;

pub fn pubFuncInBar () {}
fn prvFuncInBar () {}
\end{lstlisting}
\end{minipage}\hspace{5pt}%
\begin{minipage}[t][][t]{0.47\linewidth}
\begin{lstlisting}[caption=\textit{./foo/qux.yr}, style=coloredverbatim]
in qux;

pub fn pubFuncInQux () {}
fn prvFuncInQux () {}
\end{lstlisting}
\end{minipage}

\subsection{Describing a package}

Only one file has to be passed to the compiler. For example, considering the
file topology described in the previous example, only the \token{./foo.yr} file
is passed to the compiler. All inner modules declared in the \token{foo} module
will be reached and validated by the compiler. A file hierarchy of source files
is called a \emph{package}, and always starts with a root file.

This is the main reason for sibling modules, as only one file is the entry point
of a package, whereas sometimes multiple root modules would be needed. In this
example, the module \token{baz} is a sibling to the module \token{foo}, as it is
described within the source code in \textit{foo.yr}. The following bash command
line is used to compile the package. The \token{-fdump-ymir} option is used to
dump files representing what the compiler generated at each stage, and
\token{-nostdinc} to not automatically include the standard library and runtime.

\begin{lstlisting}[style=bashVerb, escapechar=@+]
@+\textcolor{teal!80}{alice@dev}@+:~$ gyc foo.yr -fdump-ymir -nostdinc
\end{lstlisting}

The file \token{foo.yr.ydump-decls.1} shows the symbol tree declared during the
declaration phase (before validation, so without template symbols -
\token{foo.yr.ydump-decls.2} contains all declared symbols after validation, so
including the templates).

\begin{lstlisting}[style=bashVerb, escapechar=@+]
baz
    baz::prvFuncInBaz
    pub baz::pubFuncInBaz
pub foo
    pub foo::bar
        foo::bar::prvFuncInBar
        pub foo::bar::pubFuncInBar
    foo::qux
        foo::qux::prvFunctionInQux
        pub foo::qux::pubFunctionInQux
\end{lstlisting}

\subsection{Local module}

A module can be declared directly inside another module without using another
file. In this case, the module is considered a child module, and therefore the
result is identical to a file module located in a subdirectory. For example, the
\token{foo} module described above could have been written as follows. The
generated symbol tree is exactly the same.

%\begin{minipage}{\linewidth}
\begin{lstlisting}[caption=\textit{./foo.yr}, style=coloredverbatim]
in foo;

mod baz;

pub mod bar {
  pub fn pubFuncInBar () {}
  fn prvFuncInBar () {}
}

mod qux {
  pub fn pubFuncInQux () {}
  fn prvFuncInQux () {}
}
\end{lstlisting}
%\end{minipage}

\subsection{Protection}
\label{sec:symbol_protection}

The keywords \token{pub} (for public) and \token{prv} (for private) are used
to define the protection of a symbol. By default, a symbol is private. A private
symbol is accessible only by the module that declares it and by the other child
symbols declared in the same module. In other words, a module has access to all
the symbols it has declared and all the symbols declared by its parents. But it
does not have access to the private symbols declared by its child modules, or
those declared by siblings or cousins.

To clarify the situation, let's look at the symbol lookup algorithm. This
algorithm always starts with the symbol requesting access, then looks at its
siblings, then at the parent symbol and its siblings, and then at the
grandparents and their siblings. For example, if the symbol
\token{foo::bar::pubFuncInBar} makes a symbol lookup request, then three levels
of symbols are visible as described in the figure~\ref{fig:(chap5):symbol_privacy}. In
addition to these levels, public symbols declared within visible symbols are
also visible, recursively.

\begin{center}
  \scalebox{1.3}{
    \begin{tikzpicture}

      \filldraw (4, 1) node[align=center, above] {\texttt{\tiny{baz}}};
      \draw[-] (3.9,1) -- (3,-0.2);
      \draw[-] (4.1,1) -- (5,-0.2);

      \filldraw (3, -0.6) node[align=center, above] {\texttt{\tiny{prvFuncInBaz}}};
      \filldraw (5, -0.6) node[align=center, above] {\texttt{\tiny{pubFuncInBaz}}};


      \filldraw (0, 1) node[align=center, above] {\texttt{\tiny{foo}}};
      \draw[-] (-0.1,1) -- (-1,-0.2);
      \draw[-] (0.1,1) -- (1,-0.2);

      \filldraw (1, -0.6) node[align=center, above] {\texttt{\tiny{qux}}};
      \draw[-] (0.9,-0.6) -- (0.8,-1.8);
      \draw[-] (1.1,-0.6) -- (2.1,-1.8);

      \filldraw (0.8, -2.2) node[align=center, above] {\texttt{\tiny{prvFuncInQux}}};
      \filldraw (2.4, -2.2) node[align=center, above] {\texttt{\tiny{pubFuncInQux}}};

      \filldraw (-1, -0.6) node[align=center, above] {\texttt{\tiny{bar}}};
      \draw[-] (-1.1,-0.6) -- (-2.1,-1.8);
      \draw[-] (-0.9,-0.6) -- (-0.8,-1.8);

      \filldraw (-2.4, -2.2) node[align=center, above] {\texttt{\tiny{prvFuncInBar}}};
      \filldraw (-0.8, -2.2) node[align=center, above] {\texttt{\tiny{pubFuncInBar}}};

      \draw[rounded corners, line width=0.1pt] (-3.2, -2.3) rectangle (0, -1.6);
      \filldraw (-3.2, -1.6) node[align=center, above] {\tiny {level 1}};

      \draw[rounded corners, line width=0.1pt] (-1.3, -0.7) rectangle (1.3, -0.1);
      \filldraw (-1.3, -0.1) node[align=center, above] {\tiny {level 2}};

      \draw[rounded corners, line width=0.1pt] (-0.5, 0.8) rectangle (4.5, 1.5);
      \filldraw (-0.5, 1.5) node[align=center, above] {\tiny {level 3}};

      \draw[-Stealth,shorten >=1pt] (-3.2, -1.2) to [out=90,in=180] (-1.65, 0.1);
      \draw[-Stealth,shorten >=1pt] (-1.4, 0.3) to [out=90,in=180] (-0.9, 1.7);

      \draw[-Stealth,shorten >=1pt] (1.25, -0.35) to [out=0,in=90] (2.4, -1.8);
      \draw[-Stealth,shorten >=1pt] (4.3, 1.2) to [out=0,in=90] (5.2, -0.2);

      \draw [color=black!30!green] plot [color=red, smooth, tension=0.1] coordinates {
        (-3.15,-2.3) (-0.05,-2.3) (-0, -2.25) (-0, -1.6) (0.05, -1.55) (0.1, -1.5) (1.55, -1.5) (1.6, -1.55) (1.6, -2.25) (1.65, -2.3) (3.1, -2.3) (3.15, -2.25) (3.15, -1.6)
        (3.15, -1.4) (3.1, -1.35) (2, -1.35) (2, 0.55) (2.05, 0.6)
        (4.1, 0.6) (4.15, 0.55) (4.15, -0.6) (5.8 , -0.6) (5.8, 0.2) (5, 0.2) (4.2, 1.5) (-0.2, 1.5)  (-0.5, 0.4) (-0.6, 0) (-1.2, -0.1) (-1.4, -0.4)  (-2, -1.55) (-2.05, -1.59) (-3, -1.59) (-3.1, -1.59)
        (-3.15, -1.6) (-3.2, -1.65) (-3.2, -2.2) (-3.2, -2.25) (-3.15, -2.3)
      };


    \end{tikzpicture}
  }
  \captionof{figure}{\label{fig:symbol_privacy} Example of symbol lookup protections from \texttt{foo::bar::pubFuncInBar}}
\end{center}


As a result, the symbol \token{foo::bar::pubFuncInBar} has access to the symbol
\token{foo::qux::pubFuncInQux}, even if the symbol \token{qux} is declared
private by the parent module \token{foo}. On the other hand, the symbol
\token{foo::bar::pubFuncInBar} does not have access to
\token{foo::qux::prvFuncInQux}, nor does the symbol \token{baz} have access to
the module \token{foo::qux}.

\subsection{Package importation}

As we have seen, a package is compiled by passing the path to its root file to
the compiler. But a package may depend on other packages, which need to be
imported in order to use the symbols they describe. This is done by using the
compiler's \token{-I} option to import an external package and declare its
symbols in the symbol table. All symbols imported by this method are not
validated by the compiler (except for template symbols which are generated on
invocation - see Chapter~\ref{chap:templates}). So they have to be validated
manually and linked during symbol linking.

In the following example, let's consider two packages, the package \token{foo},
located in the directory \token{/home/alice/mypackage/foo.yr}, which declares a
sub-module \token{bar}, and a second package \token{qux}, located in the path
\token{/home/alice/external/qux.yr}, and which declares a module \token{baz}.

\begin{lstlisting}[caption=\textit{/home/alice/mypackage/foo.yr}, style=coloredverbatim]
in foo;

mod bar {
  pub fn tryAccessToBaz () {
    qux::baz::funcInBaz ();
  }
}
\end{lstlisting}

\begin{lstlisting}[caption=\textit{/home/alice/external/qux.yr}, style=coloredverbatim]
in qux;

pub mod baz {
  pub fn funcInBaz () {
    std::io::println ("Success !");
  }
}
\end{lstlisting}

In order for the module \token{foo::bar} to access the module
\token{qux::baz}, the following command line must be written. It will declare
the symbol \token{qux} as a sibling module of the root module \token{foo} and
thus with the same privacy protection as described above. Therefore, the module
\token{qux::baz} must be declared public to be accessible by the module
\token{foo} and its children.

\begin{lstlisting}[style=bashVerb, escapechar=@+]
@+\textcolor{teal!80}{alice@dev}@+:~$ gyc foo.yr -I ../external/qux.yr
\end{lstlisting}

Because the \token{qux} package is not validated, the above command will result
in a linking error stating that the symbol \token{qux::baz::funcInBaz} was not
found. This is a linking error, not a validation error. To correct this error,
the \token{qux} package must first be validated and then linked when compiling
the \token{foo} package. In the following bash command lines, the option
\textit{-c} is used to create an object file (\textit{qux.o}), that is used
afterwards during the linking in the compilation of the package \textit{foo.yr}.

\begin{lstlisting}[style=bashVerb, escapechar=@+]
@+\textcolor{teal!80}{alice@dev}@+:~$ cd /home/alice/external
@+\textcolor{teal!80}{alice@dev}@+:~/external$ gyc -c qux.yr -o qux.o
@+\textcolor{teal!80}{alice@dev}@+:~/mypackage$ cd ../mypackage
@+\textcolor{teal!80}{alice@dev}@+:~/mypackage$ gyc foo.yr -I ../external/qux.yr   \
                                    ../external/qux.o
\end{lstlisting}

\mynotebox {
  The symbol \token{qux::baz::funcInBaz} has access to the symbol
  \token{std::io::println} because the compiler includes the standard library
  package by default. This import follows the same rules as any other package
  import, and this package could be imported manually using the \token{-I}
  option, followed by the path to the standard library installation, and
  \token{-nostdinc} to disable automatic import. This can be useful for testing
  new or custom versions of the standard library. For more information on the
  standard library and runtime, see Chapter~\ref{chap:std_and_core_runtime}.
}

\subsection {Using a module}

As you may have noticed, the full path of symbols has to be written to access
them, for example in \token{foo::bar:tryAccessToBaz} the access to the symbol
\token {qux::baz::funcInBaz} was in full letter. This can be cumbersome, so the
construction \token{use} was introduced. This construction is followed by the
ymir path of a module, in order to describe that a symbol name written in the
current context can come from this module. For example, let's say you need to
use the \token{println} often, then the \token{use std::io;} construction can
be useful. The use construction is enclosed in the symbol that makes the
statement and its children.

%\begin{minipage}{\linewidth}
\begin{lstlisting}[caption=\textit{./foo.yr}, style=coloredverbatim]
in foo;

mod bar {
  use qux::baz;
  use std::io;

  pub fn tryAccessToBaz () {
    funcInBaz ();
    println ("io without full name");
  }
}


fn funcInFoo () {
  std::io::println ("Need full name, foo did not use std::io");
}
\end{lstlisting}
%\end{minipage}

The declaration dump file \token{foo.yr.ydump-decls.1} contains the list of
modules used within a given symbol. The core modules are the modules that are
automatically imported and used in every module when the \token{-nostdinc}
option is not used. They contain symbols used by the runtime and referenced by
the compiler to perform high-level operations (such as deep copies of slices,
exception throwing, etc.). More information about the runtime and the standard
library is presented in Chapter~\ref{chap:std_and_core_runtime}.

\begin{lstlisting}[caption=\textit{foo.yr.ydump-decls.1}, style=bashVerb, escapechar=@+]
  pub foo - use {core, core::array, core::range,
                 core::exception, core::typeinfo,
                 core::duplication, core::math}
  foo::bar - use {core, core::array, core::range,
                  std::io, qux::baz,
                  core::exception, core::typeinfo,
                  core::duplication, core::math}
      pub foo::bar::tryAccessToBaz
  foo::funcInFoo
\end{lstlisting}

The path described in a \token{use} statement can be more complex, describing a
tree of modules to use. For example, you may want to use the modules
\token{std::io}, \token{std::fs::path}, and \token{std::fs::file}. This can
be done in a single line.

\begin{lstlisting}[style=coloredverbatim]
in foo;

use std::{io, fs::{path, file}};
\end{lstlisting}

\subsubsection*{Use statement locality}

As explained above, \token{use} statements are enclosed by the symbol that
makes them and by its children. However, in addition to this, use statements are
local to a file, meaning that child modules declared in files other than their
parent module don't have access to the use statements of their parent modules.
This is enforced to avoid cluttering up child modules with use statements they
may not need or even want, and to make use statements clearer since they're
always in the same file as the file they affect. As a result, there is a slight
difference between a child module declared locally in the same file, and a child
module declared in a subdirectory file.
\smallskip

1) When using a single file:

\begin{lstlisting}[caption=\textit{./foo.yr}, style=coloredverbatim]
in foo;
use std::io;

mod bar {
  fn inBar () {
    println("In bar."); // ok, with the use statement from /foo/.
  }
}
\end{lstlisting}

2) When using two separate files:

\hspace{-0.03\linewidth}%
\begin{minipage}[t][][t]{0.3\linewidth}%
\begin{lstlisting}[caption=\textit{./foo.yr}, style=coloredverbatim]
in foo;
use std::io;

mod bar;
\end{lstlisting}
\end{minipage}%
\hspace{0.02\linewidth}%
\begin{minipage}[t][][t]{0.65\linewidth}
\begin{lstlisting}[caption=\textit{./foo/bar.yr}, style=coloredverbatim, escapechar=@]
in bar;

fn inBar () {
  @\hb{println}@("In bar."); // error, parent use statement is hidden
}
\end{lstlisting}
\end{minipage}%

\vfill%
\pagebreak
