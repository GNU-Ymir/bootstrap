\section{Functions}%
\label{sec:functions}

A function is a piece of code that can be called to perform a defined behavior.
A function is declared using the keyword \token{fn} followed by the name of the
function to be described. A function is enclosed in a module like any other
global symbol.

\subsection{Parameters}
\label{sec:function_parameters}

The parameters of a function are declared in parentheses after the function
identifier. Parameters serve as the input of the function, allowing to pass data
to the function when calling it. A parameter declaration follows the syntax
\token{ident : T}, where \token{ident} is the parameter's identifier and
\token{T} is its type. Parameters are separated by a comma. The lifetime of the
parameters are limited to the scope of the function body. Parameters are
variables that adhere to rules that are common to standard variable
declarations, including aspects such as variable lifetime, mutability,
references, laziness, and more. These rules and specific behaviors are detailed
in Chapter~\ref{chap:variables}.

\begin{lstlisting}[style=coloredverbatim]
fn foo (a : i32, b : i32) {
  std::io::println (a, " ", b);
}
\end{lstlisting}

The specification of the parameter's type is obligatory, although its identifier
can be represented by the token \token{\_} to signify that the parameter is not
utilized by the function. Additionally, a parameter beginning and ending with
the token \token{\_} (e.g., \token{\_a\_}) is also flagged as ignorable. Further
details on handling unused variables can be found in
Section~\ref{sec:unused_variables}. While it may initially seem unnecessary,
this convention serves a purpose, particularly in function overriding scenarios
(c.f. Section~\ref{sec:function_overloading}).

\subsubsection*{Calling a function}

A function is called using the syntax \token{ident (arg1, arg2, ...)}, where
\token{ident} is the identifier of the function to call (path to the function
symbol), followed by a list of arguments enclosed in parentheses and separated
by commas. A function argument is a value passed either positionnally or by
keyword as the value to be taken by the parameter of the function when called.
To construct a keyword argument, the parameter identifier is used and associated
with a value using the \token{->} token. The following example shows the call
of the function \token{foo} using both possibilities.

\begin{lstlisting}[style=coloredverbatim]
foo (1, 2); // positional /a/, then /b/

foo (b-> 2, 1); // keyword for /b/, then position for /a/
foo (b-> 2, a-> 1); // keyword for both parameters
\end{lstlisting}

A keyword argument can be placed anywhere in the argument list. The argument
parameter assignment algorithm starts by removing the keyword arguments from the
equation and locking their assignment, and then iterates over the parameters
that do not yet have an assignment to assign them to the positional arguments.
The result of all the above calls is the same, and produce the following Ymir
Intermediate Language code (for the function \token{foo} in the module
\token{main}).

\begin{lstlisting}[style=lyilVerb]
_Y4main3fooFi32i32Zv(1, 2);
_Y4main3fooFi32i32Zv(1, 2);
_Y4main3fooFi32i32Zv(1, 2);
\end{lstlisting}

\warningbox{ the arguments of the function are constructed in the order of
  the parameters, and not in the order they appear in the function call. For
  example, if the function \token{foo} is called as described in the following
  source code, the function \token{bar} will always be called before the function
  \token{baz}. If the order of construction of the arguments is defined and
  predictible, it is not a good habit to count on it, and the use of intermediate
  variables is strongly recommanded to avoid any errors.
}

The following source code produces the Ymir Intermediate Language described in
the code block underneath. At this level, there is no difference between the
three calls, but the third one is preferred for code clarity.

\begin{lstlisting}[style=coloredverbatim]
fn bar ()-> i32 {
  println ("In bar");
  12
}

fn baz ()-> i32 {
  println ("In baz");
  15
}

foo (a-> bar (), b-> baz ());
foo (b-> baz (), bar ());
// if the above code perform the same order of operations, it is recommanded to make it clear

let x = bar ();
let y = baz (); // it is more clear that bar is called before baz
foo (a-> x, b-> y);
\end{lstlisting}
\smallskip

\begin{lstlisting}[style=lyilVerb]
// variable declarations ...
YI_1(#1) = _Y4main3barFZi32();
YI_2(#2) = _Y4main3bazFZi32();
_Y4main3fooFi32i32Zv(YI_1(#1), YI_2(#2)); // first call
YI_3(#3) = _Y4main3barFZi32();
YI_4(#4) = _Y4main3bazFZi32();
_Y4main3fooFi32i32Zv(YI_3(#3), YI_4(#4)); // second call
x(#5) = _Y4main3barFZi32();
y(#6) = _Y4main3bazFZi32();
_Y4main3fooFi32i32Zv(x(#5), y(#6)); // third call
\end{lstlisting}

\subsubsection*{Uniform call syntax}

The Uniform Call Syntax (\textit{UCS}) allows to call any function with the same
syntax as method calls (see Chapter~\ref{chap:custom_types}). The primary use of
this syntax is to chain calls and provide pipe handling to a value by placing
the first argument (associated with the first parameter) before the function
name and concatenating it with the \token{.} token. There is no need to change
the function definition to allow for \textit{UCS}.

\begin{lstlisting}[style=coloredverbatim]
fn add (a : i32, b : i32)-> i32 {
  a + b
}

let a = 12;
let b = a.add (13).add (125); // rewritten into add (add (a, 13), 125)
\end{lstlisting}

\textit{UCS} calls are simply rewritten by placing the left operand of the
binary operation \token{.} as the first positional argument of the call. It is
impossible to use a keyword argument as this left operand. The above source code
produces the following Ymir Intermediate Language result.
\smallskip

\begin{lstlisting}[style=lyilVerb]
let a(#1) : i32;
let YI_2(#2) : i32;
let b(#3) : i32;
a(#1) = 12;
YI_2(#2) = _Y4main3addFi32i32Zi32(a(#1), 13);
b(#3) = _Y4main3addFi32i32Zi32(YI_2(#2), 125);
\end{lstlisting}

\mynotebox { \textit{UCS} was introduced to perform chained operations. It can be
  compared to function composition in math (but with reverse syntax), where
  \textit{f (g (x))} is rewritten as \textit{x.g ().f ()}. An example of common
  usage is map reduce in the form \token{([1, 2, 3]).map!\{|x| => x + 1\} ().reduce!\{|x, y| => x + y\} ().}
}

\subsubsection*{Optional parameters}

A parameter within a function can have a default value, making it optional to
provide an argument when calling the function. This default value is specified
by appending the \token{=} token after the parameter type, followed by the
desired value. The optional parameter can be placed anywhere in the parameter
list.
\bigskip

\begin{lstlisting}[style=coloredverbatim]
fn foo (a : i32 = 12, b : i32) {
  std::io::println (a, " ", b);
}
\end{lstlisting}

To call the \token{foo} function described in the above code, only one argument
is needed to associate it with the \token{b} parameter. The value of the
parameter can be changed using a keyword argument.

\begin{lstlisting}[style=coloredverbatim]
foo (3); // defining parameter /b/

foo (3, a-> 18); // defining parameter /a/ and /b/
foo (a-> 18, 3); // defining parameter /a/ and /b/
\end{lstlisting}

If no other value is associated with an optional parameter, its default value is
constructed at the place of the call. This means that for a complex default
value (for example, a function call), the value is constructed before entering
the function. For example, let's look at the following code where the parameter
\token{a} has a default value constructed by calling the function \token{bar},
then the function \token{bar} is called before the function \token{foo} when
calling the function \token{foo}. The result is described in the YIL code
below.

\begin{lstlisting}[style=coloredverbatim]
fn bar ()-> i32 {
  println ("In bar.");
  12
}

fn foo (a : i32 = bar ()) {
  println ("In foo : ", a);
}

foo ();
\end{lstlisting}

\begin{lstlisting}[style=lyilVerb]
let YI_1(#1) : i32;
YI_1(#1) = _Y4main3barFZi32();
_Y4main3fooFi32Zv(YI_1(#1));
\end{lstlisting}

The default value of an optional parameter can refer to a symbol that is
accessible within the context of the function for which it is a parameter, but
because the function has not yet been entered and because it would create
complex parameter order dependencies, this value cannot refer to the other
parameters of the function.

\begin{lstlisting}[style=coloredverbatim, escapechar=@]
mod bar {
  pub fn foo (a : i32 = prvInBar ()) {
    println (a);
  }

  fn prvInBar ()-> i32 {
    12
  }
}

bar::foo (); // ok, no need to have access to prvInBar from here because foo has the access

@\hb{bar::prvInBar}@ (); // error, prvInBar is private within this context
\end{lstlisting}

\subsubsection*{Recursive optional parameter}

An optional parameter can be constructed by a function, it can even be
constructed by calling the function in which it is a parameter. However, since
simply calling the function without changing the value of this default parameter
would result in an infinite recursion, in such contexts default values might be
disabled and made mandatory. For example, in the following source code the
validation of the function \token{foo} fails because the parameter \token{a}
is self-dependent. For the same reason, the functions \token{bar} and
\token{baz} are not allowed.

\begin{lstlisting}[style=coloredverbatim]
fn foo (a : i32 = foo ())-> i32 {
  a + 1
}

fn bar (a : i32 = baz ())-> i32 {
  a + 1
}

fn baz (a : i32 = bar ())-> i32 {
  a - 1
}
\end{lstlisting}

To solve this problem, the value of the parameter \token{a} can be set inside
the recursive calls. For example, the following code shows a solution for the
functions \token{bar} and \token{baz}. This fix may seem abrupt, and one could
argue that only one of the two functions needs to stop the infinite recursion.
However, it was decided to force this, as it seems to be a fairly niche problem
that needs to be avoided anyway.

\begin{lstlisting}[style=coloredverbatim]
// setting the option parameter a in baz, to stop the infinite recursion
fn bar (a : i32 = baz (a-> 1))-> i32 {
  a + 1
}

// setting the option parameter a in bar, to stop the infinite recursion
fn baz (a : i32 = bar (a-> 1))-> i32 {
  a - 1
}

bar ();
\end{lstlisting}

The above source code produce the following YIL result.

\begin{lstlisting}[style=lyilVerb]
let YI_1(#1) : i32;
let YI_2(#2) : i32;
YI_1(#1) = _Y4main3bazFi32Zi32(1);
YI_2(#2) = _Y4main3barFi32Zi32(YI_1(#1));
\end{lstlisting}

\subsection {Body, expressions and statements}
\label{sec:function_body}

A function body is an expression that describe a value. There are several types
of expressions in Ymir, some of which have already been described in previous
chapters, such as literal values, binary expressions, etc. A common expression
that is generally used to describe the body of a function is a block. A block is
a list of expressions separated by semicolons and enclosed in curly braces. The
last expression contained in a block may or may not end with a semicolon, 1) if
not, then the value of the block is described by that last expression, 2)
otherwise the block value is a unit value of type \token{void}.

\begin{lstlisting}[style=coloredverbatim]
fn foo (a : i32)-> i32
  a + 1 // body of /foo/

fn bar ()-> i32
{ // start of a block
  let a = 12;
  let b = 34;
  a + b // value of the block
}

fn baz ()-> void {
  let a = 12;
  let b = 34;
  std::io::println (a + b);
} // block value is <unit>

\end{lstlisting}

We can distinguish two kinds of expressions, those that have a value and those
that don't (or actually have the value \token{unit} of type \token{void}). An
expression that has no value is called a statement. Ending an expression with a
semicolon inside a block construction turns it into a statement.

\begin{lstlisting}[style=coloredverbatim, escapechar=@]
fn foo () {
  let a = @\hb{(let b = 12)}@; // error, /let b = 12/ is of type void

  let y = {
    let x = 1;
    x + 1
  };
}
\end{lstlisting}

In the above example at line 2, the statement \token{let b = 12;} does not have
a value, so it cannot be used as the initial value in the variable declaration
of \token{a}. On line 4, however, the expression \token{\{let x = 1; x + 1\}}
has a value and is evaluated to the value \token{2} of type \token{i32}, so it
can be used to describe the initial value of \token{y}. All expressions and
statements are listed in Chapter~\ref{chap:type_and_values}.

\subsubsection*{Empty function}

A function with no body defines a function that is described externally. This
feature is is used to provide header only module files, where source code is
hidden but still callable externally. The actual code of the function has to be
linked at compilation. Let's suppose the two packages \token{foo} and
\token{bar}, where the full source code of the package \token{bar} is
described in \token{bar/src/bar.yr} and a header file was created to be
imported by the package \token{foo} in \token{bar/include/bar.yri}. A
\textit{yri} file described a header file that has been automatically generated
by the compiler (cf. Section\ref{sec:external_decls}).


\begin{lstlisting}[caption=\textit{./bar/src/bar.yr}, style=coloredverbatim]
in bar;
use std::io;

pub fn funcInBar (a : i32)-> i32 {
  println ("Hello from bar");
  a + 1
}
\end{lstlisting}

%\begin{minipage}[t][][t]{0.47\linewidth}
\begin{lstlisting}[caption=\textit{./bar/include/bar.yri}, style=coloredverbatim]
in bar;

pub fn funcInBar (a : i32)-> i32;
\end{lstlisting}
%% \end{minipage}\hspace{5pt}
%% \begin{minipage}[t][][t]{0.47\linewidth}
\begin{lstlisting}[caption=\textit{foo.yr}, style=coloredverbatim]
in foo;

use bar;
use std::io;

pub fn main () {
  println ("Hello from foo");
  println ("Res : ", funcInBar (12));
}
\end{lstlisting}
%\end{minipage}


The file \token{./bar/include/bar.yri} can be published alonside with the
binary file of the \token{bar} package, that way the source code of the package
\token{bar} is hidden. It has the same purpose as header files that could be
found in C/C++ languages.

\begin{lstlisting}[style=bashVerb, escapechar=@+]
@+\textcolor{teal!80}{alice@dev}@+:~$ cd bar/src/
@+\textcolor{teal!80}{alice@dev}@+:~/bar/src$ gyc -c bar.yr -o ../../libbar.o
@+\textcolor{teal!80}{alice@dev}@+:~/bar/src$ cd ../../
@+\textcolor{teal!80}{alice@dev}@+:~$ gyc foo.yr -I bar/include/bar.yri libbar.o -o foo.exe
@+\textcolor{teal!80}{alice@dev}@+:~$ ./foo.exe
Hello from foo
Hello from bar
Res : 13
\end{lstlisting}

\subsection {Return value}

The return type of a function is described at the end of the prototype of a
function declaration, after the parameters, with the syntax \token{-> T}, where
\token{T} is the type returned by the function. This information can be omitted
if the function does not return a value and is therefore of type \token{void}.
The body of the function must produce a value of the same type as that defined
in its prototype.

\begin{lstlisting}[style=coloredverbatim]
fn foo ()-> i32 {
  12
}
fn bar () {}
fn baz ()-> void {}
\end{lstlisting}

The keyword \token{return} can be used to return early, it is associated with a
value written after the keyword, defining the value to return. The type of this
value must be the same as the type of the function.

\begin{lstlisting}[style=coloredverbatim]
fn foo ()-> i32 {
  return 12;
}

fn bar () {
  return; // return with /unit/ value
}
\end{lstlisting}

The value of the function is retreived from the caller of the function, as the
result of the call expression.

\begin{lstlisting}[style=coloredverbatim]
let resultOfFoo = foo ();

// It can also be used inside another expression, as any expression
let add = foo () + 12;
\end{lstlisting}

\subsubsection*{Early exits}
\label{sec:function_early_return}

In case of early return, the type of the body of the function must be
\token{void} if all the branches are leading to an early exit statement. Any
statement and expressions that are not atteignable because they are written
after an early exit are not allowed.

\begin{lstlisting}[style=coloredverbatim, escapechar=@]
fn foo (b : bool)-> i32 {
  if (b) return 12;
  else return 0;

  @\hb{println ("Unreachable !");}@// error, unreachable statement
  @\hb{87}@// unreachable as well
}
\end{lstlisting}

There are three other ways to exit a function early, 1) throw an exception (see
the next subsection), 2) fail an assertion, or 3) panic. All of these early
exits are taken into account by the compiler to compute the exit nodes of the
flow graph and send unreachable statement errors.

\subsection {Exceptions}

A function can exit through various methods, returning normaly and taking the
value of the body, by an early return and taking the value associated with the
return statement, by panicking or by throwing exceptions. The last type of early
exit makes the function unsafe as it exit the function without a value, and
therefore has to be taken into account in the caller function.

It is mandatory to inform in the prototype of the function that it can exit
without returning a value, by using the syntax \token{throws T, U, ...} after
the return type of the function, where \token{T} is a type of exceptions, class
inheriting from the core type \token{core::exception::Exception}. This syntax
lists all the types of exception that can be thrown by the function. Inside the
body of the function the statement \token{throw V;}, with \token{V} a
exception value, exits the function early by throwing an exception.

\begin{lstlisting}[style=coloredverbatim]
fn foo (a : bool)-> i32
  throws AssertError
{
  println ("In foo.");
  if (!a) throw AssertError::new ("a must be true."); // exit the function

  println ("Exiting foo.");
  22 // return the value /22/
}
\end{lstlisting}

When calling the function \token{foo}, thanks to that information, it become
apparent inside the caller that \token{foo} is throwing exception and might
return without a value. The compiler enforce the caller function to take that
into account either by catching the exception, and perform other operations in
case of the failure of the function \token{foo} (this will be treated in
Chapter~\ref{chap:control_flows}), or by enforcing the caller function to also
define in its prototype that exception might be thrown.

\begin{lstlisting}[style=coloredverbatim]
fn bar ()
   throws AssertError
{
  println ("In bar.");
  let x = foo (false) + 32; // /foo/ can throw /AssertError/
  println ("Result of bar : ", x);
}
\end{lstlisting}

Exception might not only exit the function that is throwing the exceptions, but
also the caller functions that are not catching it. If no function catches the
exception (going through all functions until the main function), the program
panics. However, it is easy to avoid program panic due to exception throwing as
it would be apparent in the prototype of the \token{main} function, that the
program might throw an exception. When no exception are displayed in the
prototype of the \token{main} function, then the program cannot panic due to
exception throwing.

In the next code block (calling the \token{bar} function displayed in the
previous code block), the exception is caught. The result of the full program is
displayed in the result block underneath.

\mytipbox {
  In the main function, the call to \token{eprintln} has the same prototype
  as the function \token{println}, but write to stderr instead of stdout.
}

\begin{lstlisting}[style=coloredverbatim]
fn main () {
  {
    bar ();
  } catch {
    _ => { // catching all exceptions
      eprintln ("An error occured in bar");
    }
  }
  println ("Normally exiting program.");
}
\end{lstlisting}

\mycautionbox { Exceptions should primarily be used for error handling, not for
  routine value checking in control flows. Option values are preferable for
  representing the absence of a value in such scenarios, promoting clearer and
  faster code. }

More information about error handling is presented in
Chapter~\ref{chap:control_flows}. In that chapter will be presented the close
relation between error handling using option values and exception throwing and
how to pass from one to another using Ymir syntax and simple code.

\subsection {Unsafe function}

A function using unsafe without defining an unsafe context has to be tagged as
unsafe. This is done by putting a tag before the prototype of the function using
the syntax \token{@unsafe}.

\importantbox{ This is not perfect, because a function that is not marked as
  unsafe may still perform an unsafe operation, but within an unsafe block
  (e.g., in the next code, the call to \token{baz} is considered safe, but
  it isn't). The same issue exists with panics. There has been an attempt to
  introduce an effect system where functions that use unsafe or panic must
  always declare it, but it has been put aside for the moment. It is difficult
  to design such a system without greatly over-complexifying the code. However,
  it may come back in future versions, in a form close to the Effekt language. }


You cannot just mark all functions that use an unsafe block as unsafe, because
all functions will end up being considered unsafe (many functions from cores and
the standard library use unsafe code that has been verified by hand but cannot
be proven), and just one unsafe function will pollute all callers recursively
back to the main function. In the next example, you can see that the function
\token{baz} is not marked as unsafe, but calls the function \token{foo}, which
uses pointer dereferencing. This is an unsafe operation; in practice, due to the
nature of pointers, there is no way to check the validity of a pointer either at
compile time or at run time, and thus catch an unsafe behavior to make it safe.
Hence, the \token{bar} function is actually unsafe. There is not much we can do
about this, but at least the unsafe behavior is explicitly marked up in the
source code, which reduces the work involved in detecting unsafe code.

\begin{lstlisting}[style=coloredverbatim, escapechar=$]
@unsafe
fn foo () {
  let mut a = 12;
  let dmut b = &a;

  *b = 34; // unsafe, but not in unsafe block
}

fn bar () {
  $\hb{foo ();}$ // error, foo is unsafe but /bar/ isn't
}

fn baz () {
  unsafe { // ok, as foo is unsafe
    foo ();
  }
}
\end{lstlisting}

\subsection{The main function}

A program always starts with a function called \token{main}. This function has
a fixed prototype that must be respected. It can take either no parameter or a
parameter of type \token{[[c8]]}. If a parameter is defined, it takes as value
the list of command line options that were passed to the program when it was
started (the runtime manages the transformation of the C-like parameters
\token{(int, char**)} into a slice of strings).

\begin{lstlisting}[style=coloredverbatim, caption=The simplest main function prototype]
fn main () {
  println ("Hello world!");
}
\end{lstlisting}

The function can either return a value of type \token{i32} or no value
(\textit{unit} value of type \token{void}). It can throw exceptions, i.e. the
program may end in an error when this exception is thrown because it is not
caught by any function. It can also be unsafe.

\begin{lstlisting}[style=coloredverbatim, caption=The most complex main function prototype]
use std::{io, conv}; // for string to int conversion

@unsafe
fn main (args : [[c8]])-> i32
   throws AssertError
{
  println ("Command line options : ", args);
  if (args.len != 2us) throw AssertError::new ("Missing or too much paramters");

  let mut a = args [1].to!{u32} ();
  let dmut b = &a;

  // useless unsafe code, to make the main function unsafe :)
  println ("First parameter is : ", *b);

  *b + 12 // return value of the program
}
\end{lstlisting}

\mynotebox {
  The \token{main} function is mandatory to create a runnable program, in the
  previous source code examples a lot of code may appear to be written at the root
  level of a file, but that was just for verbosity. All control flow is described
  inside functions, and the first function called is the \token{main} function.
  Only one main function can be described in a program, it can be in any module.
}

\vfill%
\pagebreak
