\section{Global variable}%
\label{sec:global_variables}

A global variable is declared as a global symbol (within a module) using the
keyword \token{lazy} followed by an identifier. Global variables are thoroughly
presented in Chapter~\ref{chap:variables}.

\begin{lstlisting}[style=coloredverbatim]
lazy A = 12;

// Global variable protection system is the same as any global symbol
pub lazy B = A;
\end{lstlisting}

Global variables in Ymir are inherently lazy, meaning they are instantiated and
constructed only upon their first reference at runtime. This design addresses
the complexity arising from global variables referencing each other across
different modules. Constructing a dependency graph of symbols to ensure that
dependent global variables are created in the correct order would require a
sophisticated system. Such a system would be extremely complex to build and
sometimes impossible, as cycles could exist without being visible at compile
time (e.g., global variables from external packages). Hence, the lazy
instantiation approach simplifies this process. Laziness ensures that when a
global variable references another global variable, the referenced variable is
necessarily constructed.

\begin{lstlisting}[style=coloredverbatim]
lazy A = foo (); lazy B = bar (A);

fn foo ()-> i32 {
  println ("In foo");
  42
}

fn bar (a : i32)-> i32 {
  println ("In bar")
  a * 2
}

fn main () {
  println ("In main");
  println ("B = ", B); // first reference to B in the program
  println ("A = ", A); // second reference to A in the program
}
\end{lstlisting}

In the above listing, the construction of the global variable \token{B} depends
on the value of the global variable \token{A}. In other C-like languages (C++,
D, etc.), this would be problematic as there would be no guarantee that
\token{A} would be constructed before \token{B}. Thanks to the lazy system, when
the variable \token{B} is constructed, it triggers the construction of the
variable \token{A}. Therefore, \token{A} always has a value when \token{B} is
being constructed. The result of the above source code is presented in the next
listing.

\begin{lstlisting}[style=bashVerb, escapechar=@+]
In main
In foo
In bar
B = 84
A = 42
\end{lstlisting}

\vfill%
\pagebreak
