\section{A first program}

A Ymir program is composed of symbols of different type. The first type of
symbols that one needs to know when writting their first program is function.
Indeed, a ymir program is always composed of at least one function that is
called when the program is executed. This function is named \token{main}. In the
following listing, a simple program that write \textit{Hello World!} to the
console is described. This program is composed of the function \token{main}
introduced by the token \token {fn}. This main function calls the function
\token{println}, which is described in the standard library, and used
to display texts to the console. The parentheses following the \token{println}
function name, are used to list parameters to pass to the function, here a
string literal. Note that statements are followed by a semicolon. The first line
of the example program starts with the keyword \token{use}. This line is used to
describe that a function is being used from the \token{std::io} module (here the
function \token{println}), which is automatically imported by the compiler since
it belongs to the standard library. After calling the \token{println}
function, the main function is exited and with it the program.

\begin{lstlisting}[style=coloredverbatim, caption=Source file \textit{hello.yr}]
use std::io;

fn main () {
  println ("Hello World!");
}
\end{lstlisting}


\subsection{Compilation and execution}

Once we have a source file, the compiler has to be invoked to compile it and
create the executable file. The following listing presents the command line to
execute in order to produce the \textit{hello} executable file, using the
\textit{hello.yr} source file.

\begin{lstlisting}[style=bashVerb, escapechar=@+]
@+\textcolor{teal!80}{alice@dev}@+:~$ gyc hello.yr -o hello
\end{lstlisting}

Once the compiler has finished, a new file named \textit{hello} will be present
in the current directory. This file is executable, and its execution will
display the text \textit{Hello World!}.

\begin{lstlisting}[style=bashVerb, escapechar=@+]
@+\textcolor{teal!80}{alice@dev}@+:~$ ./hello
Hello World!
\end{lstlisting}





