\section{Gyllir}

Gyllir is the build system and package manager for the Ymir language. Although
the ymir language can be used without Gyllir, it is a useful tool that handles a
lot of work for programmers, such as creating proper command lines to build the
code, managing project versions, downloading library dependencies, etc. In this
book, the Gyllir tool is used to manage the source code, and it is assumed that
you will use it as well. Gyllir can initialize a new project with the
\textit{init} subcommand. This command asks some questions (with default values)
to create an empty project and a default configuration.

\begin{lstlisting}[style=bashVerb, escapechar=@+]
@+\textcolor{teal!80}{alice@dev}@+:~$ mkdir my-project
@+\textcolor{teal!80}{alice@dev}@+:~$ cd my-project
@+\textcolor{teal!80}{alice@dev}@+:~@+\textcolor{green!60!black}{/my-project}@+$ gyllir init
Name [main]: my-project
Author name [Alice Doe <alice-doe@mail.com>]:
Description [A minimal Ymir app]:
License [proprietary]:
Type (executable/library) [executable]:
Registry [local:/home/alice/.local/gyllir/my-project]:
\end{lstlisting}

Once executed, there will be several files in the current directory.
The \textit{gyllir.toml} file contains the configuration of the gyllir project
and is read by the gyllir command to specify the behavior of the tool when
invoked. In this default configuration, the tool will build an executable when
the \textit{gyllir build} command is invoked. The directory \textit{src}
contains all source files of the project, here a single file \textit{main.yr}
with a placeholder content. The directory \textit{test} contains a source file
for unit tests. These tests will be discussed in another chapter.

\begin{lstlisting}[style=bashVerb, escapechar=@+]
@+\textcolor{teal!80}{alice@dev}@+:~@+\textcolor{green!60!black}{/my-project}@+$ tree
@+\color{teal}{.}@+
├── gyllir.toml
├── @+\color{teal}{src}@+
│   └── main.yr
└── @+\color{teal}{test}@+
    └── __test__.yr

3 directories, 3 files
\end{lstlisting}

The command \textit{gyllir build} produces an executable file. The
command \textit{gyllir run} produces the same executable, but runs it
immediately.

\begin{lstlisting}[style=bashVerb, escapechar=@+]
@+\textcolor{teal!80}{alice@dev}@+:~@+\textcolor{green!60!black}{/my-project}@+$ gyllir run
  @+\color{black!30!applegreen}{Compiling}@+ my-project v0.1.0
  @+\color{black!30!applegreen}{Compiling}@+ 1 modules in thread 1 (among 1 changed files of 1)
  @+\color{black!30!applegreen}{Produced}@+ executable file my-project
  @+\color{black!30!applegreen}{Executing}@+ project my-project v0.1.0

Hello World !
\end{lstlisting}

\vfill
\pagebreak
