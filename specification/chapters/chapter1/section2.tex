\section{Structure of a simple Ymir program}

A ymir program can be composed of multiple symbols, for example multiple
functions. In the next example, three functions are described \token{main},
\token{foo} and \token{bar}. There is no need to define the symbols in a
particular order, however it is more common to put the \token{main} function at
the last position in the source file (or even put the main function in a
separate module).

\begin{lstlisting}[style=coloredverbatim, caption=First structured source file]
use std::io;

fn foo () {
  println ("Called foo");
  bar (); // calling the bar function
}

fn bar () {
  // because we use std::io, println can be called directly
  println ("Called bar");
}

/**
 * The entry point of the program
 */
fn main () {
  println ("Program started");
  foo ();
}
\end{lstlisting}

This example introduces the notion of comments. Comments are part of a source
file that are ignored by the compiler, and are only meant to be read by a human
and help them understand the code they are reading. We will see in a future
chapter, that comments can also be used to generate code documentation
automatically. A comments can be introduced using two different syntaxes:
\begin{itemize}
\item Everything following the token \token{//} until the end of the line, to
  create a single line comment.
\item Everyting enclosed within \token{/*} and \token{*/}, to create a multiline comment.
\end{itemize}

\subsection{Source code layout}

The compiler parses a source file by breaking its content into tokens, such as
parentheses, comas, brackets, etc. Tokens can be separated by an arbitry number
of white characters, that is spaces, line returns and tabulations. For example,
a function can be written in a single line, or by breaking into multiple lines
following a specific layout.


\begin{lstlisting}[style=coloredverbatim, caption=Arbitrary code layout example]
fn foo(){println("Called foo");bar();}

fn
main
(   )
{
println ("Program started");
foo();
}
\end{lstlisting}

\subsection{Prefered layout}

Because a strange layout can make the source code difficult to read, a standard
layout is generally prefered. In this layout, only one statement must be defined
in a line, and statements must be aligned according to their scope depth (a
scope being defined by the tokens \token{\{\}}). Spaces surrounds the operators,
and parameter operators such as \token{()} and \token{[]} are always spaced as
following \token{foo (a, b, c)}.

All the source codes presented in this book respect this layout, and it is
strongly encouraged to use the same layout when writting ymir programs, in order
to keep everything uniform.
